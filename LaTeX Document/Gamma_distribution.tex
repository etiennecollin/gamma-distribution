\input{~/Programming/LaTex/Paths}
\path{macOS}

\documentclass[12pt]{article}
\usepackage[english]{babel}																												% Selects language of document
\usepackage{./_assets/Packages}																											% Imports custom packages selection
\usepackage[style=numeric,autocite=superscript,sorting=none,backref=false,backend=biber,date=iso,datezeros=true,seconds=true]{biblatex}	% Bibliography options. autocite=<inline, footnote, superscript, plain>

% \usepackage{indentfirst}																												% Used to indent first paragraph of section

%======================================================================================================================
% DOCUMENT USER SETTINGS ==============================================================================================
%======================================================================================================================
\newcommand{\docAuthorName}{Etienne Collin, Marie Ouellet \& Rania Yahyaoui}
\newcommand{\docAuthorStudentNumber}{}
\newcommand{\docAuthorTitlePage}{\docAuthorName}
\newcommand{\docClass}{Probability and Statistics}
\newcommand{\docClassInstructor}{Professor Vincent Carrier}
\newcommand{\docClassNumber}{201-BNM-LW}
\newcommand{\docClassTime}{Section 20108}
\newcommand{\docClassSemester}{Winter 2022}
\newcommand{\docDueDate}{May 13, 2022}
\newcommand{\docDueTime}{23:59}
\newcommand{\docSubTitle}{Properties, Proofs and Applications}
\newcommand{\docTitle}{The Gamma Distribution}
\usepackage[svgnames, hyperref]{xcolor}	% See p.42 https://mirror.its.dal.ca/ctan/macros/latex/contrib/xcolor/xcolor.pdf

% Page settings
\topmargin=-0.45in
\evensidemargin=0in
\oddsidemargin=0in
\textwidth=6.5in
\textheight=9.0in
\headsep=0.25in
\linespread{1.1}
\pagestyle{fancy}
\fancyhf[HL]{\firstxmark}						% Content on left of head
\fancyhf[HC]{\docTitle}						% Content on center of head
\fancyhf[HR]{Collin – \docAuthorStudentNumber}			% Content on right of head
\fancyhf[FL]{\docClassInstructor}					% Content on left of foot
\fancyhf[FC]{\thepage}						% Content on center of foot
\fancyhf[FR]{\lastxmark}						% Content on right of foot
\renewcommand\headrulewidth{0.4pt}				% Width of line separation head from body
\renewcommand\footrulewidth{0.4pt}				% Width of line separation foot from body
% \setlength\parindent{5mm}					% Custom Paragraph indentation
\setlength\parindent{0.5in}					% Custom Paragraph indentation											% Imports custom page settings
% DEFINITIONS =========================================================================================================
% Paths
\addbibresource{\basepath/bibliography.bib}	% Imports bibliography file
\graphicspath{{images/}}

% lstlistings setup ---------------------------------------------------------------------------------------------------
\definecolor{commentsColor}{gray}{0.6}
\definecolor{keywordsColor}{RGB}{249,38,114}
\definecolor{stringColor}{RGB}{230, 219, 116}
\definecolor{mygreen}{rgb}{0,0.6,0}
\definecolor{mygray}{rgb}{0.5,0.5,0.5}
\definecolor{mymauve}{rgb}{0.58,0,0.82}
\lstset{
  backgroundcolor=\color{white},   % choose the background color; you must add \usepackage{color} or \usepackage{xcolor}; should come as last argument
  basicstyle=\footnotesize,        % the size of the fonts that are used for the code
  breakatwhitespace=false,         % sets if automatic breaks should only happen at whitespace
  breaklines=true,                 % sets automatic line breaking
  captionpos=b,                    % sets the caption-position to bottom
  commentstyle=\color{mygreen},    % comment style
  deletekeywords={},            % if you want to delete keywords from the given language
  escapeinside={\%*}{*)},          % if you want to add LaTeX within your code
  extendedchars=true,              % lets you use non-ASCII characters; for 8-bits encodings only, does not work with UTF-8
  firstnumber=0,                % start line enumeration with line 0
  frame=single,	                   % adds a frame around the code
  keepspaces=true,                 % keeps spaces in text, useful for keeping indentation of code (possibly needs columns=flexible)
  keywordstyle=\color{blue},       % keyword style
  language=Python,                 % the language of the code
  morekeywords={*},            % if you want to add more keywords to the set
  numbers=left,                    % where to put the line-numbers; possible values are (none, left, right)
  numbersep=5pt,                   % how far the line-numbers are from the code
  numberstyle=\tiny\color{mygray}, % the style that is used for the line-numbers
  rulecolor=\color{black},         % if not set, the frame-color may be changed on line-breaks within not-black text (e.g. comments (green here))
  showspaces=false,                % show spaces everywhere adding particular underscores; it overrides 'showstringspaces'
  showstringspaces=false,          % underline spaces within strings only
  showtabs=false,                  % show tabs within strings adding particular underscores
  stepnumber=2,                    % the step between two line-numbers. If it's 1, each line will be numbered
  stringstyle=\color{mymauve},     % string literal style
  tabsize=4,	                   % sets default tabsize to 2 spaces
  title=\lstname                   % show the filename of files included with \lstinputlisting; also try caption instead of title
}

% Sections alignment --------------------------------------------------------------------------------------------------
% Gets label of current section
\makeatletter\newcommand{\currentlabel}{\@currentlabelname}\makeatother
% Sets alignment of sections
\newcommand{\formatsection}[1]{
    \ifthenelse{\equal{#1}{center}}{
		%Makes sections centered
		\titleformat{\section}{\normalfont\Large\bfseries\centering}{\thesection}{1em}{}
    }{
        \ifthenelse{\equal{#1}{right}}{
			%Makes sections right-aligned
			\titleformat{\section}{\normalfont\Large\bfseries\raggedleft}{\thesection}{1em}{}
        }{
			\ifthenelse{\equal{#1}{left}}{
				%Makes sections left-aligned
				\titleformat{\section}{\normalfont\Large\bfseries\raggedright}{\thesection}{1em}{}
        	}{
        	}
        }
    }
}

% tabularx
\newcolumntype{Y}{>{\centering\arraybackslash}X}

% Caption/Ref settings -------------------------------------------------------------------------------------------------
\captionsetup{
    format = hang,
    labelfont = sc,
	skip = 6pt,
	hypcapspace = 96pt
}

% TOC & Sections -------------------------------------------------------------------------------------------------------
% Sets TOC depth and change changes its display name
\setcounter{secnumdepth}{3} % Change to 0 for unnumbered sections
\addto\captionsenglish{
  \renewcommand{\contentsname}{Table of Contents}
}
% Makes the paragraph command appear like a subsubsubsection
\titleformat{\paragraph}{\normalfont\normalsize\bfseries}{\theparagraph}{1em}{}
\titlespacing*{\paragraph}{0pt}{3.25ex plus 1ex minus .2ex}{1.5ex plus .2ex}

% SIunitx -------------------------------------------------------------------------------------------------------------
% Sets \per to display as frac
\sisetup{per-mode=symbol-or-fraction}
\DeclareSIUnit\Molar{\mole\per\liter}
\DeclareSIUnit\molar{\textsc{M}}
\DeclareSIUnit\clight{\text{\ensuremath{c}}_{0}}

% BibLaTex ------------------------------------------------------------------------------------------------------------
% BibLaTex definitions
\let\oldcite\cite
\renewcommand{\cite}{\autocite}
\let\oldsupercite\supercite
\renewcommand{\supercite}[1]{\textsuperscript{[}\oldsupercite{#1}\textsuperscript{]}}
% Makes number created by \supercite to be between brackets.
\DeclareCiteCommand{\supercite}[\mkbibsuperscript]{%
	\iffieldundef{prenote}{}{\BibliographyWarning{Ignoring prenote argument}}%
	\iffieldundef{postnote}{}{\BibliographyWarning{Ignoring postnote argument}}%
}
{\usebibmacro{citeindex}\,\bibopenbracket\usebibmacro{cite}\bibclosebracket}
{\supercitedelim}{}

% BASIC DOCUMENT SETTINGS AND TITLE PAGE ==============================================================================
% PDF export settings
\hypersetup{
	pdftex,
	pdfauthor = {Etienne Collin},
	bookmarksnumbered = true,
	bookmarksopen = true,
	bookmarksopenlevel = 1,
	pdfstartview = Fit,
	pdfpagemode = UseOutlines,
	pdfpagelayout = TwoPageRight,
	colorlinks = true,		% Colors links instead of ugly boxes
	urlcolor = DarkBlue,	% Color for external hyperlinks
	linkcolor = black,		% Color of internal links
	citecolor = black,		% Color of citations
	filecolor = black,		% Color of file links
}

% Custom font

\newcommand{\customfont}[1]{
	\usepackage{mathspec}	% Enable the power of CUSTOM FONTS with XeLaTeX
	\setallmainfonts(Digits,Latin,Greek)[Numbers={Lining,Proportional}]{#1}
	\setallmonofonts[Numbers={Lining,Proportional}]{#1}
	\setallsansfonts[Numbers={Lining,Proportional}]{#1}
	\setmathsfont(Digits,Latin,Greek)[Numbers={Lining,Proportional}]{#1}
	\setmathsf[Numbers={Lining,Proportional}]{#1}
	\setmathbb[Numbers={Lining,Proportional}]{#1}
	\setmathcal[Numbers={Lining,Proportional}]{#1}
}

% ENVIRONMENTS ========================================================================================================
% hmwkProblem ---------------------------------------------------------------------------------------------------------
\newcounter{partCounter}
\renewcommand{\part}[1]{\noindent\textbf{\large Part (\alph{partCounter})}\stepcounter{partCounter}\\}
\newcounter{hmwkProblemCounter}
\setcounter{hmwkProblemCounter}{1}

% Continued Header Changer
\newcommand{\enterProblemHeader}[1]{
	\nobreak\extramarks{}{}\nobreak{}
	\nobreak\extramarks{Problem \arabic{#1} (continued)}{Problem \arabic{#1} continued on next page\ldots}\nobreak{}
}
\newcommand{\exitProblemHeader}[1]{
	\nobreak\extramarks{Problem \arabic{#1} (continued)}{Problem \arabic{#1} continued on next page\ldots}\nobreak{}
	\stepcounter{#1}
	\nobreak\extramarks{}{}\nobreak{}
}

% Environment. Optional argument will adjust the problem counter.
\newenvironment{hmwkProblem}[1][-1]{
	\ifnum#1>0
		\setcounter{hmwkProblemCounter}{#1}
	\fi
	\pagebreak
	\section{Problem \arabic{hmwkProblemCounter}}
	\stepcounter{section}
	\setcounter{partCounter}{1}
	\enterProblemHeader{hmwkProblemCounter}
}{
	\exitProblemHeader{hmwkProblemCounter}
}

% sect ----------------------------------------------------------------------------------------------------------------
% Will write "continued" when content spans >1 page. Argument adjusts the name of the section.
\newenvironment{sect}[1]{
	\pagebreak
	\section{#1}
	\stepcounter{section}
	\setcounter{partCounter}{1}
	\nobreak\extramarks{}{}\nobreak{}
	\nobreak\extramarks{\currentlabel\ (continued)}{\currentlabel\ continued on next page\ldots}\nobreak{}
}{
	\nobreak\extramarks{\currentlabel\ (continued)}{\currentlabel\ continued on next page\ldots}\nobreak{}
	\nobreak\extramarks{}{}\nobreak{}
}

% SETUP/ALIASES =======================================================================================================
\newcommand{\imagecenter}[2][1]{
	\begin{center}
		\includegraphics[width=\textwidth*{#1}]{#2}
	\end{center}
}
\newcommand{\imageleft}[2][1]{
	\begin{flushleft}
		\includegraphics[width=\textwidth*{#1}]{#2}
	\end{flushleft}
}
\newcommand{\imageright}[2][1]{
	\begin{flushright}
		\includegraphics[width=\textwidth*{#1}]{#2}
	\end{flushright}
}
\newcommand{\figurecenter}[4]{
	\begin{figure}[H]
		\centering
		\begin{minipage}[H]{#1\textwidth}
			\centering
			\includegraphics[width=\textwidth]{#2}
			\caption{#3}
			\label{#4}
		\end{minipage}
	\end{figure}
}
\newcommand{\figuredual}[6]{
	\begin{figure}[H]
		\begin{minipage}[H]{0.48\textwidth}
			\centering
			\includegraphics[width=\textwidth]{#1}
			\raggedleft
			\caption{#2}
			\label{#3}
		\end{minipage}
		\hfill
		\begin{minipage}[H]{0.48\textwidth}
			\centering
			\includegraphics[width=\textwidth]{#4}
			\caption{#5}
			\label{#6}
		\end{minipage}
	\end{figure}
}
\newcommand{\figuredualsingle}[6]{
	\begin{figure}[H]
		\begin{minipage}[H]{0.48\textwidth}
			\centering
			\includegraphics[width=\textwidth]{#1}
			\subcaption{#2}\label{#6-a}
		\end{minipage}
		\hfill
		\begin{minipage}[H]{0.48\textwidth}
			\centering
			\includegraphics[width=\textwidth]{#3}
			\subcaption{#4}\label{#6-b}
		\end{minipage}
		\caption{#5}\label{#6}
	\end{figure}
}

% Theorems
\theoremstyle{plain}
\newtheorem{definition}{Definition}[subsection]
\newtheorem{theorem}{Theorem}[subsection]
\newtheorem{example}{Example}[theorem]
\newtheorem{corollary}{Corollary}[subsection]
\newtheorem{lemma}{Lemma}[theorem]
\newtheorem{proposition}{Proposition}[theorem]
% Unnumbered version
\newtheorem*{definition*}{Definition}
\newtheorem*{theorem*}{Theorem}
\newtheorem*{example*}{Example}
\newtheorem*{corollary*}{Corollary}
\newtheorem*{lemma*}{Lemma}
\newtheorem*{proposition*}{Proposition}

\theoremstyle{definition}
\newtheorem{exercise}{Exercise}[subsection]
% Unnumbered version
\newtheorem*{exercise*}{Exercise}


\theoremstyle{remark}
\newtheorem{remark}{Remark}
\newtheorem{case}{Case}
\newtheorem{claim}{Claim}[subsection]
% Unnumbered version
\newtheorem*{remark*}{Remark}
\newtheorem*{case*}{Case}
\newtheorem*{claim*}{Claim}

\renewcommand\qedsymbol{QED}

% Useful shortcuts
\newcommand{\alg}[1]{\textsc{\bfseries\footnotesize #1}}
\newcommand{\deriv}[2][]{\mathop{}\!\frac{\mathrm{d}^{#1}}{\mathrm{d}#2^{#1}}}	% For derivatives
\newcommand{\pderiv}[3][]{\mathop{}\!\dfrac{\partial^{#1}#2}{\partial#3^{#1}}}					% For partial derivatives
\newcommand{\pdiff}[2][]{\mathop{}\!\dfrac{\partial^{#1}}{\partial#2^{#1}}}					% For partial derivatives
\newcommand{\diff}[2][]{\mathop{}\!\mathrm{d^{#1}}#2}							% Integral dx
\newcommand*{\widefbox}[1]{\fbox{\hspace{2em}#1\hspace{2em}}}						% For boxed align environment

% Enables new list environment that supports nesting
\newlist{legal}{enumerate}{10}
\setlist[legal]{label*=.\arabic*}
\setlist[legal,1]{label=\arabic*}

% Temporary text markers
\newcommand{\solution}{\noindent\textbf{\large Solution}}				% Alias for the Solution section header
\newcommand{\missing}{\textcolor{red}{\textbf{VALUE MISSING}}}	% Creates a big red "Missing Value" string.
\newcommand{\inprogress}{\textcolor{red}{\textbf{IN PROGRESS}}}	% Creates a big red "In Progress" string.
\newcommand{\incomplete}{\textcolor{red}{\textbf{INCOMPLETE}}}	% Creates a big red "Incompletes" string.
\newcommand{\bigred}[1]{\textcolor{red}{\textbf{#1}}}	% Creates a big red string.

% END OF SETUP DOCUMENT ===============================================================================================											% Imports custom environments and definitions
% \customfont{Comic Sans MS}													% Use a custom font

\fancyhf[HR]{\docClassTime}													% Removes student number from right header
\newcommand{\B}{\mathcal{B}}
\newcommand{\E}{\mathcal{E}}
\newcommand{\G}{\mathcal{G}}
\newcommand{\J}{\mathcal{J}}
\newcommand{\N}{\mathcal{N}}


%======================================================================================================================
% SOURCE ==============================================================================================================
%======================================================================================================================
\begin{document}
\singlespacing
% Title page
\begin{titlepage}
	\begin{center}
		\vspace*{0.5in}
		\LARGE{\textsc{\docTitle}}\\
		\vspace*{0.1in}
		\Large{\textsc{\docSubTitle}}\\
		\vfill
		%
		\Large{\textsc{\docAuthorTitlePage}}\\
		\vspace*{0.1in}
		\Large{\textsc{\docClass\ –\ \docClassNumber}}\\
		\vfill
		%
		\Large{\docClassTime}\\
		\vspace*{0.1in}
		\large{\textsc{\docClassInstructor}}\\
		\vfill
		%
		\large{\textsc{St. Lawrence Cégep Champlain}}\\
		\vspace*{0.1in}
		\small{Due on \docDueDate\ at \docDueTime}\\
		\vfill
	\end{center}
\end{titlepage}
\pagebreak
\phantomsection\tableofcontents\pagebreak
\doublespacing
%======================================================================================================================
% START OF DOCUMENT ===================================================================================================
%======================================================================================================================
\section{Introduction}\vspace*{-24pt}
\subsection{Historical Context \& Parameterization}
The gamma distribution\cite{carrierProbabilityStatistics201BNM05} is a distribution of continuous random variables $X$
with two nonnegative parameters $\alpha$ and $\lambda$. A gamma distributed random variable $X$ is represented as:
\vspace*{-24pt}
\begin{equation*}
	X\sim\G\text{amma}(\alpha, \lambda)
\end{equation*}\\[-40pt]
or, alternatively, as:\vspace*{-6pt}
\begin{equation*}
	X\sim\Gamma(\alpha,\lambda)
\end{equation*}\\[-60pt]

The gamma probability distribution gave rise to many other functions and distributions, such as the gamma function, the
exponential distribution, the Erlang distribution or the chi-square distribution. Originally, it was first presented by
Leon Hard Euler, a Swiss mathematician and physician. As Euler gained popularity, the gamma distribution was further
researched by important figures in the domain of mathematics such as Karl Weierstrass, Carl Friedrich Gauss, Charles
Hermite, and many more\cite{hoschGammaDistribution2017, wikipediaGammaDistribution2022,
sebahIntroductionGammaFunction2002}.

The gamma distribution is part of the two-parameter family of continuous probability distributions. Indeed, it may be
parameterized with two different parameterizations\cite{wikipediaGammaDistribution2022}:

\noindent Parameterization 1:\vspace*{-24pt}
\begin{align*}
	\text{Shape: } \alpha>0		&&	\text{Rate: }\lambda>0
\end{align*}\\[-60pt]

\noindent Parameterization 2:\vspace*{-24pt}
\begin{align*}
	\text{Shape: } k>0			&&	\text{Scale: }\theta>0
\end{align*}\\[-60pt]

In the second parameterization, the scale parameter $\theta$ corresponds to the inverse of the rate parameter $\lambda$,
allowing flexibility in the modelization of the distribution. Indeed, the application of the scale parameter is
pertinent when one wants to use the value of the mean time between events in a Poisson process, as
opposed to the mean rate of occurrence during one unit of time.

That being said, in this document, as the two parameterizations only exist for the sake of convenience and are identical
in their results, only parameterization 1 will be considered and used for proofs.

%======================================================================================================================
%======================================================================================================================
%======================================================================================================================

\pagebreak
\subsection{Probability Density Function}\label{subsec:gamma:pdf}
The probability density function of a gamma distributed random variable $X$ corresponds to

\begin{equation}\label{eq:gamma:pdf}
	f(x) = \frac{\lambda^\alpha}{\Gamma(\alpha)}x^{\alpha-1}e^{-\lambda x}\text{ for }x\in (0, \infty)
\end{equation}

In statistics, it is often used to model the time between independent events that occur at a constant average rate in a
Poisson process. Indeed, the parameter $\alpha$ indicates the number of events that are modelled, and the parameter
$\lambda$ specifies the average rate at which they occur. Here is a graph of the probability density function of a gamma
distribution with parameters $\alpha=4$ and $\lambda=3$. As an example, this probability density function could be used
to evaluate the time until four events that occur at a rate of three events per unit time take place:
\figurecenter{0.9}{Gamma_PDF.png}{Plot of the probability density function of a gamma distribution.}{fig:graph:gamma:pdf}

Let's now prove that this equation is a probability density function. That is, let's show that:
\begin{equation}
	\int_{D_X} f(x)\diff{x} = 1
\end{equation}
\begin{proof}
	Let's use the following substitution to solve the integral:
	\begin{empheq}[box=\widefbox]{align*}
		u=\lambda x;	&&	\diff{u}=\lambda\diff{x}
	\end{empheq}
	\begin{equation}
		\begin{split}
			\int_{D_X} f(x)\diff{x}	&=	\int^\infty_0\frac{\lambda^\alpha}{\Gamma(\alpha)}x^{\alpha -1}e^{-\lambda x}\diff{x}\\
									&=	\frac{1}{\lambda}\int^\infty_0\frac{\lambda^\alpha}{\Gamma(\alpha)}\left(\frac{u}{\lambda}\right)^{\alpha-1}e^{-u}\diff{u}\\
									&=	\frac{1}{\Gamma(\alpha)}\int^\infty_0 u^{\alpha-1}e^{-u}\diff{u}\\
									&=	1
		\end{split}
	\end{equation}
\end{proof}
It is now clear that \autoref{eq:gamma:pdf} is a probability density function.

%======================================================================================================================
%======================================================================================================================
%======================================================================================================================

\pagebreak
\subsection{Cumulative Distribution Function}\label{subsec:gamma:cdf} The cumulative distribution
function (CDF) of the gamma distribution is defined as
\begin{equation}
	F(x) = \int^x_0\frac{\lambda^\alpha}{\Gamma(\alpha)}u^{\alpha-1}e^{-\lambda u}\diff{u}
\end{equation}
However, this integral has no definite form, that is, the integrand has no antiderivative. Hence, this CDF is often
represented in its simplified form using the lower incomplete gamma function\cite{weissteinIncompleteGammaFunction}:
\begin{equation}
	\gamma(\alpha, \lambda x)\equiv\int^x_0u^{\alpha-1}e^{-u}\diff{u}
\end{equation}
such that
\begin{equation}
	F(x) = \frac{\gamma(\alpha, \lambda x)}{\Gamma(\alpha)}
\end{equation}
Interestingly, there exists a special case of the CDF: when $\alpha$ is a positive integer, that is, when the distribution
is an Erlang distribution (see \autoref{subsec:specialcases:erlang}), the CDF follows the following series expansion
\cite{wikipediaGammaDistribution2022}:
\begin{equation}
	F(x) = 1-\sum^{\alpha-1}_{i=0}\frac{(\lambda x)^i}{i!}e^{-\lambda x} = e^{-\lambda x}\sum^\infty_{i=\alpha}\frac{(\lambda x)^i}{i!}
\end{equation}
That being said, generally, although lacking a closed form, the cumulative distribution function of the gamma
distribution may be graphed using software:
\figurecenter{1}{Gamma_CDF.png}{Plot of the cumulative distribution function of a gamma distribution.}{fig:graph:gamma:cdf}

%======================================================================================================================
%======================================================================================================================
%======================================================================================================================

\pagebreak
\section{Properties}
Now that the fundamentals of the gamma distribution are set, a few properties stemming from the probability density
function and the cumulative distribution function will be presented and proven. These properties, along
with being extremely intriguing, are essential to multiple computations and statistical analyses.

\subsection{Moment Generating Function}\label{subsec:gamma:mgf}
The moment generating function (MGF) of a real number, vector or matrix random variable $X$ is used to compute a
distribution's moment. It provides an extremely useful tool for distribution analysis and proofs related to many other
properties. Unlike the characteristic function, which will be briefly explored in \autoref{subec:advancedproperties:cf},
the MGF does not always exist for real values of a distribution \cite{wikipediaMomentgeneratingFunction2022}. Here is
the MGF of the gamma distribution:
\begin{equation}
	M(t)=\left(\frac{\lambda}{\lambda-t}\right)^\alpha\text{ for }t<\lambda
\end{equation}
This function may be found using the fact that $M(t)=E(e^{tx})$. Hence:
\begin{equation}
	\begin{split}
		M(t)	&=	E(e^{tx})\\
				&=	\int^\infty_0 e^{tx}f(x)\diff{x}\\
				&=	\int^\infty_0 e^{tx}\frac{\lambda^\alpha x^{\alpha-1}}{\Gamma(\alpha)}e^{-\lambda x}\diff{x}\\
				&=	\int^\infty_0 \frac{\lambda^\alpha x^{\alpha-1}}{\Gamma(\alpha)}e^{-(\lambda-t) x}\diff{x}\\
				&=	\frac{\lambda^\alpha}{(\lambda-t)^\alpha}\int^\infty_0 \frac{(\lambda-t)^\alpha x^{\alpha-1}}{\Gamma(\alpha)}e^{-(\lambda-t) x}\diff{x}\\
				&=	\frac{\lambda^\alpha}{(\lambda-t)^\alpha}(1)\\
				&=	\left(\frac{\lambda}{\lambda-t}\right)^\alpha
	\end{split}
\end{equation}

%======================================================================================================================
%======================================================================================================================
%======================================================================================================================

\pagebreak
\subsection{Expected Value}\label{subsec:gamma:expected}
\subsubsection{Finding the Equation}
The expected value, also known as the theoretical mean, may be found for the gamma distribution using the following
formula:
\begin{equation}
	\mu = E(X) = \frac{\alpha}{\lambda}
\end{equation}
Interestingly, there are multiple ways to prove this result. Three different proofs will be explored in this paper.

\paragraph{Proof 1}\label{par:gamma:expected:proof1}
The first technique uses the gamma distribution probability density function presented in \autoref{subsec:gamma:pdf} as
a means to get the expected value. In fact:
\begin{equation}\label{eq:gamma:expected:proof1}
	E(X) = \int^\infty_0 xf(x)\diff{x} = \frac{\alpha}{\lambda}
\end{equation}
\begin{proof}
	\begin{equation}
		\begin{split}
			E(X)	&=	\int^\infty_0 xf(x)\diff{x}\\
					&=	\int^\infty_0 \frac{\lambda^\alpha}{\Gamma(\alpha)}x^\alpha e^{-\lambda x}\diff{x}\\
					&=	\int^\infty_0 \frac{\lambda^\alpha}{\Gamma(\alpha)}x^{(\alpha+1)-1} e^{-\lambda x}\diff{x}\\
					&=	\frac{\Gamma(\alpha+1)}{\lambda\Gamma(\alpha)}\int^\infty_0 \frac{\lambda^{\alpha+1}}{\Gamma(\alpha+1)}x^{(\alpha+1)-1} e^{-\lambda x}\diff{x}\\
					&=	\frac{\alpha\Gamma(\alpha)}{\lambda\Gamma(\alpha)}(1)\\
					&=	\frac{\alpha}{\lambda}
		\end{split}
	\end{equation}
\end{proof}

\paragraph{Proof 2}
Using the moment generating function obtained in \autoref{subsec:gamma:pdf}, the expected value of a gamma distribution
may be found. That is:
\begin{equation}
	E(X) = M^\prime(0) = \frac{\alpha}{\lambda}
\end{equation}
\begin{proof}
	\begin{equation}\label{eq:gamma:mgfprime}
		\begin{split}
			M^\prime(t)	&=	\deriv{t}\left(\frac{\lambda}{\lambda - t}\right)^\alpha\\
						&=	\lambda\alpha\left(\frac{\lambda}{\lambda - t}\right)^{\alpha-1}\deriv{t}\left(\frac{1}{\lambda - t}\right)\\
						&=	\frac{\lambda\alpha}{(\lambda - t)^2}\left(\frac{\lambda}{\lambda - t}\right)^{\alpha-1}
		\end{split}
	\end{equation}
	\begin{equation}
		\begin{split}
			M^\prime(0)	&=	\frac{\lambda\alpha}{\lambda^2}\left(\frac{\lambda}{\lambda}\right)^{\alpha-1}\\
						&=	\frac{\alpha}{\lambda}\\
		\end{split}
	\end{equation}
\end{proof}

\paragraph{Proof 3}
A third way to demonstrate that $E(X) = \frac{\alpha}{\lambda}$, is by performing the following integral:
\begin{equation}
	\int^\infty_0 1-F(x)\diff{x}
\end{equation}
However, this technique is not possible in the case of the gamma distribution, as there is no explicit formula for $F(x)$
as seen in \autoref{subsec:gamma:cdf}.

\subsubsection{Special Case}
The expected value may be used to derive several other properties. For example, consider i.i.d random variable
$X\sim\G\text{amma}(\alpha, \lambda)$ such that $f(x) = \frac{\lambda^\alpha}{\Gamma(\alpha)}x^{\alpha-1}e^{-\lambda x}$.
Then:
\begin{equation}
	\begin{split}
		E(X^\beta)	&=	\int^\infty_0 x^\beta f(x)\diff{x}\\
					&=	\int^\infty_0 x^\beta \frac{\lambda^\alpha}{\Gamma(\alpha)}x^{\alpha-1}e^{-\lambda x}\diff{x}\\
					&=	\frac{\lambda^\alpha}{\Gamma(\alpha)}\int^\infty_0 x^{(\alpha+\beta)-1}e^{-\lambda x}\diff{x}\\
					&=	\frac{\lambda^\alpha}{\Gamma(\alpha)}\frac{\Gamma(\alpha+\beta)}{\lambda^{\alpha+\beta}}\int^\infty_0 \frac{\lambda^{\alpha+\beta}}{\Gamma(\alpha+\beta)}x^{(\alpha+\beta)}\diff{x}\\
					&=	\frac{\Gamma(\alpha+\beta)}{\lambda^\beta \Gamma(\alpha)}\text{ for }\beta\in (-\alpha, \infty)
	\end{split}
\end{equation}
The expected value and the variance can be determined using the provided formula above:
\paragraph{Expected Value}
If $E(X^\beta) = \frac{\Gamma(\alpha+\beta)}{\lambda^\beta \Gamma(\alpha)}$, then:
\begin{equation}
	\begin{split}
		E(X^1)	&=	\frac{\Gamma(\alpha+1)}{\lambda^1 \Gamma(\alpha)}\\
		E(X^1)	&=	\frac{\alpha\Gamma(\alpha)}{\lambda\Gamma(\alpha)}\\
				&=	\frac{\alpha}{\lambda}
	\end{split}
\end{equation}

\paragraph{Variance}
In order to find the variance, $E(X^2)$ will need to be found using the same technique used to determine the formula for
$E(X)$. If $E(X^\beta) = \frac{\Gamma(\alpha+\beta)}{\lambda^\beta \Gamma(\alpha)}$, then:
\begin{equation}
	\begin{split}
		E(X^2)	&=	\frac{\Gamma(\alpha+2)}{\lambda^2 \Gamma(\alpha)}\\
				&=	\frac{\alpha(\alpha+1)\Gamma(\alpha)}{\lambda^2 \Gamma(\alpha)}\\
				&=	\frac{\alpha^2+\alpha}{\lambda^2}
	\end{split}
\end{equation}
The variance can now be found with this technique:
\begin{equation}
	\begin{split}
		\text{Var}(X)	&=	E(X^2) - [E(X)]^2\\
						&=	\frac{\alpha^2+\alpha}{\lambda^2} - \left[\frac{\alpha}{\lambda}\right]^2\\
						&=	\frac{\alpha^2+\alpha}{\lambda^2} - \frac{\alpha^2}{\lambda^2}\\
						&=	\frac{\alpha}{\lambda^2}
	\end{split}
\end{equation}
In the next section, we will examine other techniques that may be used to get the variance of the gamma distribution.

%======================================================================================================================
%======================================================================================================================
%======================================================================================================================

\pagebreak
\subsection{Variance}\label{subsec:gamma:variance}
The variance is a measure of how "spread out" the data is. Indeed, the lower the variance is, the sharper the
probability density function graph will be. More specifically, the variance is the square of the standard
deviation of the random variable examined\cite{wikipediaVariance2022}.

This property may be calculated using the following equation:
\begin{equation}\label{eq:gamma:var}
	\text{Var}(X) = \frac{\alpha}{\lambda^2}
\end{equation}
and this last equation may be proven in two different ways using either the moment generating function or an integral.

\paragraph{Proof 1}
This first proof, featuring the moment generating function obtained in \autoref{subsec:gamma:mgf}, will prove the
following relationship.
\begin{equation}\label{eq:gamma:var:proof1}
	\text{Var}(X) = M^{\prime\prime}(0) - [M^\prime(0)]^2 = \frac{\alpha}{\lambda^2}
\end{equation}
\begin{proof}
	Consider the following result from \autoref{eq:gamma:mgfprime} in \autoref{subsec:gamma:expected}:
	\begin{equation}
		M^\prime(t) = \frac{\lambda\alpha}{(\lambda - t)^2}\left(\frac{\lambda}{\lambda - t}\right)^{\alpha-1}
	\end{equation}
	Then $M^{\prime\prime}(t)$ is derived to be:\vspace*{-12pt}
	\begin{equation}
		\begin{split}
			M^{\prime\prime}(t)	&=	\deriv{t}M^\prime(t)\\
								&=	\deriv{t}\left[\frac{\lambda\alpha}{(\lambda - t)^2}\left(\frac{\lambda}{\lambda - t}\right)^{\alpha-1}\right]\\
								&=	\lambda\alpha\left(\deriv{t}\left[\frac{1}{(\lambda - t)^2}\right]\left(\frac{\lambda}{\lambda - t}\right)^{\alpha-1} + \left(\frac{1}{(\lambda - t)^2}\right)\deriv{t}\left[\left(\frac{\lambda}{\lambda - t}\right)^{\alpha-1}\right]\right)\\
								&=	\lambda\alpha\left(\left(\frac{2}{(\lambda-t)^3}\right)\left(\frac{\lambda}{\lambda - t}\right)^{\alpha-1} + \left(\frac{\lambda(\alpha-1)}{(\lambda - t)^2}\right)\left(\frac{\lambda}{\lambda-t}\right)^{\alpha-2}\deriv{t}\left[\frac{1}{\lambda-t}\right]\right)\\
								&=	\lambda\alpha\left(\left(\frac{2}{(\lambda-t)^3}\right)\left(\frac{\lambda}{\lambda - t}\right)^{\alpha-1} + \left(\frac{\lambda(\alpha-1)}{(\lambda - t)^2}\right)\left(\frac{\lambda}{\lambda-t}\right)^{\alpha-2}\left(\frac{1}{(\lambda-t)^2}\right)\right)\\
								&=	\lambda\alpha\left(\left(\frac{2}{(\lambda-t)^3}\right)\left(\frac{\lambda}{\lambda - t}\right)^{\alpha-1} + \left(\frac{\lambda(\alpha-1)}{(\lambda - t)^4}\right)\left(\frac{\lambda}{\lambda-t}\right)^{\alpha-2}\right)
		\end{split}
	\end{equation}
	Let's not even try to simplify further and directly substitute $t=0$:\vspace*{-12pt}
	\begin{equation}
		\begin{split}
			M^{\prime\prime}(0)	&=	\lambda\alpha\left(\left(\frac{2}{(\lambda-0)^3}\right)\left(\frac{\lambda}{\lambda - 0}\right)^{\alpha-1} + \left(\frac{\lambda(\alpha-1)}{(\lambda - 0)^4}\right)\left(\frac{\lambda}{\lambda-0}\right)^{\alpha-2}\right)\\
								&=	\lambda\alpha\left(\frac{2}{\lambda^3}(1) + \frac{\lambda(\alpha-1)}{\lambda^4}(1)\right)\\
								&=	\lambda\alpha\left(\frac{2}{\lambda^3} + \frac{\alpha-1}{\lambda^3}\right)\\
								&=	\lambda\alpha\left(\frac{\alpha+1}{\lambda^3}\right)\\
								&=	\frac{\alpha^2 + \alpha}{\lambda^2}
		\end{split}
	\end{equation}\\[-30pt]
	Finally:\vspace*{-12pt}
	\begin{equation}
		\begin{split}
			\text{Var}(X)	&=	 M^{\prime\prime}(0) - [M^\prime(0)]^2\\
							&=	\frac{\alpha(\alpha+1)}{\lambda^2} - \left(\frac{\alpha}{\lambda}\right)^2\\
							&=	\frac{\alpha^2+\alpha}{\lambda^2} - \frac{\alpha^2}{\lambda^2}\\
							&=	\frac{\alpha^2 + \alpha -\alpha^2}{\lambda^2}\\
							&=	\frac{\alpha}{\lambda^2}
		\end{split}
	\end{equation}
\end{proof}

\paragraph{Proof 2}
This time, the proof features an integral to obtain the variance. That is:
\begin{equation}\label{eq:gamma:var:proof2}
	\text{Var}(X) = E(X^2)-[E(X)]^2 = \frac{\alpha}{\lambda^2}
\end{equation}
\begin{proof}
	In the following equations, $E(X)$ is the expected value obtained in \autoref{subsec:gamma:expected} and $E(X^2)$ is
	obtained using a process similar to \autoref{eq:gamma:expected:proof1} of \nameref{par:gamma:expected:proof1} in
	\autoref{subsec:gamma:expected}. That is:
	\begin{equation}
		\begin{split}
			E(X^2)	&=	\int^\infty_0 x^2f(x)\diff{x}\\
					&=	\int^\infty_0 \frac{\lambda^\alpha}{\Gamma(\alpha)}x^{\alpha+1} e^{-\lambda x}\diff{x}\\
					&=	\int^\infty_0 \frac{\lambda^\alpha}{\Gamma(\alpha)}x^{(\alpha+2)-1} e^{-\lambda x}\diff{x}\\
					&=	\frac{\Gamma(\alpha+2)}{\lambda^2\Gamma(\alpha)} \int^\infty_0 \frac{\lambda^{\alpha+2}}{\Gamma(\alpha+2)}x^{(\alpha+2)-1} e^{-\lambda x}\diff{x}\\
					&=	\frac{\alpha(\alpha+1)\Gamma(\alpha)}{\lambda^2\Gamma(\alpha)}(1)\\
					&=	\frac{\alpha(\alpha+1)}{\lambda^2}
		\end{split}
	\end{equation}
	Then,
	\begin{equation}
		\begin{split}
			\text{Var}(X)	&=	E(X^2)-[E(X)]^2\\
							&=	\frac{\alpha(\alpha+1)}{\lambda^2} - \left(\frac{\alpha}{\lambda}\right)^2\\
							&=	\frac{\alpha^2+\alpha}{\lambda^2} - \frac{\alpha^2}{\lambda^2}\\
							&=	\frac{\alpha^2 + \alpha -\alpha^2}{\lambda^2}\\
							&=	\frac{\alpha}{\lambda^2}
		\end{split}
	\end{equation}
\end{proof}
Conclusively, \autoref{eq:gamma:var} is confirmed.

%======================================================================================================================
%======================================================================================================================
%======================================================================================================================

\pagebreak
\subsection{Quantiles}
Usually, the quantile function is defined as the inverse of the cumulative distribution function. Since the gamma
distribution has no closed form for its CDF (see \autoref{subsec:gamma:cdf}) and, therefore, the latter cannot be
inverted, it has no quantile function. In other words, there is no explicit form for the quantile function:
\begin{equation}
	q(p)=F^{-1}(p)
\end{equation}
because $F^{-1}(x)$ is not defined, since $f(x)$ has no antiderivative.

Visually, quantiles may be represented as equal subdivisions of a probability density function graph starting from the
median. That is, in the following figure, the first quantile is represented by the section weighing 34\% of the whole
PDF, the second one represents the section weighing 13.5\% and the last one represents the section weighing 2\%. As may
be noticed, these subdivisions are equally divided both sides of the median.
\figurecenter{1}{Quantiles.png}{Visual representation of quantiles in a probability density distribution. Provided by GeeksforGeeks \cite{geeksforgeeksQuantileQuantilePlots2021}}{fig:graph:quantiles}


%======================================================================================================================
%======================================================================================================================
%======================================================================================================================

\pagebreak
\subsection{Mode}
The mode is defined as the value appearing the most through a certain set of data values, that is, the location of the
highest peak on a probability density function plot. On \autoref{fig:graph:gamma:mode}, the mode represents the value of
$T$ where the density is at its peak.
\figurecenter{1}{Gamma_mode.png}{Plot of the probability density function of three distinct gamma distributions with different modes.}{fig:graph:gamma:mode}
As may be observed in the plot of $\G\text{amma}(1, 0.3)$, the highest value of the density on the curve is $0.3$ at
$T=0$ which corresponds to the mode of this specific parameterization of the distribution. The mode of the density
function of $\G\text{amma}(1, 0.3)$ is thus $0.3$. To be more precise and avoid the need for a graph, the following
equation may be used to calculate the mode of a gamma distribution:
\begin{equation}\label{eq:gamma:mode}
	\text{Mode} = \frac{(\alpha-1)}{\lambda}\text{ for }\alpha \geq 1
\end{equation}
Ergo, for the plot of $\G\text{amma}(5, 3)$, the mode can be determined to be:
\begin{equation}
	T = \frac{(5-1)}{3} = \frac{4}{3}
\end{equation}
Accordingly, on \autoref{fig:graph:gamma:mode}, the highest peak of the plot of $\G\text{amma}(5, 3)$ is at
$\frac{4}{3}$.

\subsubsection{Special Cases}
There are two special cases concerning the mode of the gamma distribution: when the density of the PDF goes to infinity
(the graph has a vertical asymptote) and when the parameter $\alpha=1$.

\paragraph{Case 1:}
The first special case is when the mode of the PDF never reaches its maximum value, that is, when there is a vertical asymptote
on its plot. The plot of $\G\text{amma}(0.5, 1)$ on \autoref{fig:graph:gamma:mode} features this case. Indeed, if $f(x)$
is the probability density function, then:\vspace*{-20pt}
\begin{equation}
	\lim_{T\to 0^+}f(x)=\infty
\end{equation}\\[-40pt]
Notice how this case arises when the condition stated in \autoref{eq:gamma:mode} is broken, that is, that $\alpha\geq
1$. In fact, in the case of the purple plot, $\alpha=0.5<1$
\paragraph{Case 2:}
In a gamma distribution $\G\text{amma}(1, \lambda)\equiv\E\text{xp}(\lambda)$, the highest point of the plot of the PDF
will be located at $T=0$ and will correspond to a density value of $\lambda$. In \autoref{fig:graph:gamma:mode}, the
plot of $\G\text{amma}(1, 0.3)$ is an example of this special case as it has $\alpha=1$ as a parameter. Hence, the mode
of this latter parameterization of the gamma distribution is of $\lambda=0.3$ at $T=0$.

%======================================================================================================================
%======================================================================================================================
%======================================================================================================================

\pagebreak
\subsection{Inflection Points}
The more the value of alpha increases, the narrower the density function graph of $\G\text{amma}(\alpha,\lambda)$
becomes narrower. For example, when $\alpha=1$, the density function graph does not contain any inflection point.
However, its $y$-intercept is equal to $\lambda$. As the value of $\alpha$ increases to 5, two inflection points, on
each side of the mode, can be found on the graph. Inflection points are obtained by performing the second derivative of
the curve. Below are found different possible shapes of the density function graph in function of the value of $\alpha$.
By increasing the value of $\alpha$, the inflection points become closer to each other.
\figurecenter{1}{Gamma_inflection.png}{Plot of the probability density function of four distinct gamma distributions with different inflection points.}{fig:graph:gamma:inflection}

%======================================================================================================================
%======================================================================================================================
%======================================================================================================================

\pagebreak
\subsection{Median}
The median of a probability distribution is the value that separates the area under the curve of its distribution
function in two equal parts. It follows that, in the case of the gamma distribution, the median is defined as the value
$m$ for which:
\begin{equation}
	\frac{1}{2} = \frac{\lambda^\alpha}{\Gamma(\alpha)} \int^m_0 x^{\alpha-1} e^{-\lambda x}\diff{x}
\end{equation}

However, it is interesting to note that the median of the gamma distribution has no simple closed form : due to its
highly varying shape, it cannot be represented in an equation as a function of its parameters. Interestingly, some have
noticed that an approximation of the median can be performed in cases where the parameter
$\lambda=1$\cite{wikipediaGammaDistribution2022}:
\begin{equation}\label{eq:gamma:median:approximation}
	m = \alpha(1-\frac{1}{9\alpha})
\end{equation}

Let's take the case of a distribution $\G\text{amma}(5, 1)$ to examine this claim. The approximation of the median using
\autoref{eq:gamma:median:approximation} is as follows:
\begin{equation}
	\begin{split}
		m	&=	\alpha(1-\frac{1}{9\alpha})\\
			&=	5(1-\frac{1}{9\cdot 5})\\
			&=	\frac{44}{9}\\
			&=	4.\bar{8}
	\end{split}
\end{equation}

Let's confirm the validity of the answer using the graph of the probability density function of the distribution
$\G\text{amma}(5, 1)$.
\figurecenter{1}{Gamma_median_approximation.png}{Plot of the probability density function of a gamma distribution with parameters $\alpha=5$ and $\lambda=1$.}{fig:graph:gamma:median:approximation}
The approximation seems to, in fact, be valid as the PDF is "weighted" more towards the right and its median seems to be
between 4.5 and 5.

Still, there is comfort in the fact that some special cases of the gamma distribution do have a
median, such as the exponential distribution. Indeed, the median of a continuous random variable following the gamma
distribution such that $X\sim\G\text{amma}(1, \lambda)\equiv\E\text{xp}(\lambda)$ may be obtained through a simple
integration process:
\begin{equation}
	\begin{split}
		\frac{1}{2}					&=	\frac{\lambda^\alpha}{\Gamma(\alpha)} \int^m_0 x^{\alpha-1} e^{-\lambda x}\diff{x}\\
		\Rightarrow\frac{1}{2}		&=	-e^{-\lambda m} + 1\\
		\Rightarrow\frac{1}{2}		&=	e^{-\lambda m}\\
		\Rightarrow\ln(\frac{1}{2})	&=	-\lambda m\\
		\Rightarrow-\ln(2)			&=	-\lambda m\\
		\Rightarrow m				&=	\frac{\ln(2)}{\lambda}
	\end{split}
\end{equation}
Hence, the median of an exponential distribution is defined as $\frac{\ln(2)}{\lambda}$.

%======================================================================================================================
%======================================================================================================================
%======================================================================================================================

\pagebreak
\subsection{Memoryless Property}
The memoryless property indicates that a given probability distribution is independent of its history: the likelihood of
something happening in the future has no relation to past events. In statistics, a continuous random variable $X$
satisfies the memoryless property if:
 \begin{equation}
	P(X>x+a | X>a) = P(X>x)\text{ for }a, x \geq 0
 \end{equation}

 It is easy to show that the exponential distribution (see \autoref{subsec:specialcases:exponential}) displays the
 memoryless property.
 \begin{equation}
	 \begin{split}
		 P(X>x+a | X>a)	&=	\frac{P(X>x+a, X>a)}{P(X>a)}\\
		 				&=	\frac{P(X>x+a)}{P(X>a)}\\
						&=	\frac{1-F_X(x+a)}{1-F_X(a)}\\
						&=	\frac{e^{-\lambda(x+a)}}{e^{-\lambda a}}\\
						&=	e^{-\lambda x}\\
						&=	P(X>x)
	 \end{split}
 \end{equation}
As a consequence, in a sequence of exponentially distributed events, the expected time until the next event always
corresponds to $\frac{1}{\lambda}$, no matter how much time has passed since the last event.

However, it is interesting to note that when $\alpha>1$, the future behaviour of the probability distribution is not
independent of its past. Therefore, in general, the gamma distribution does not possess the memoryless property. This
can be understood intuitively when looking at a case involving $\G\text{amma}(100, 1)$, which models the total time
until 100 events separated by an average waiting time of 1 minute occur. It can be shown by brute force that if one has
already waited $x=110$ minutes, the probability of having to wait at least another ten minutes for the hundredth event
to occur is much lower than the probability of having to wait more than ten minutes in total\cite{myersCS547Lecture,
probabilitycourseExponentialDistribution}.
\begin{equation}
	\begin{split}
		P(X>110+10 | X>110)	&=			\frac{P(X>110 | X>120)\cdot P(X>120)}{P(X>110)}\\
							&\approx	0.1760\\
		P(X>10)				&\approx	1\\
		P(X>110+10 | X>110)	&\neq		P(X>10)
	\end{split}
\end{equation}
Clearly, the memoryless property is not satisfied.

%======================================================================================================================
%======================================================================================================================
%======================================================================================================================

\pagebreak
\section{The Gamma Function}
\vspace*{-6pt}
Studied by Leonhard Euler (1707-1783), the gamma function is related to the gamma distribution. This extremely handy
function allows the generalization of the factorial notation. As a matter of fact, while the factorial $\alpha!$ is
restricted to $\alpha\in\mathbb{N}$, the gamma function $\Gamma(\alpha)$ allows the computation of fractional and
negative numbers such that $\alpha\in\mathbb{Q}$. By abusing the notation, the gamma function may even be used to
compute $\Gamma(\alpha)$ such that $\alpha\in\mathbb{R}$. The relationship between the gamma function and the factorial
notation is as follows:
\vspace*{-18pt}
\begin{equation}\label{eq:gammafunction:factorial}
	\Gamma(\alpha)=(\alpha-1)!\text{ for }\alpha\in\mathbb{N}
\end{equation}
\vspace*{-60pt}

\subsection{Definition}\label{subsec:gammafunction:definition}
\vspace*{-6pt}
First of all, the gamma function is defined as
\vspace*{-12pt}
\begin{equation}
	\Gamma(\alpha)=\int^\infty_0 x^{\alpha-1}e^{-x}\diff{x} \textnormal{ for }\alpha\in\mathbb{R}^+
\end{equation}\\[-36pt]
Assuming that $\alpha > 1$, the closed form of the gamma function may be found with integration by parts:\\[-12pt]
\begin{empheq}[box=\widefbox]{align*}
	u=x^{\alpha-1};								&&	\diff{v}=e^{-x}\diff{x}\\
	\diff{u}=(\alpha-1)x^{\alpha-2}\diff{x};	&&	v=-e^{-x}
\end{empheq}
\begin{equation}
	\begin{split}
		\Gamma(\alpha)	&=	\int^\infty_0 x^{\alpha-1}e^{-x}\diff{x}\\
						&=	-\left[x^{\alpha-1}e^{-x}\right]^\infty_0 + (\alpha-1)\int^\infty_0 x^{\alpha-2}e^{-x}\diff{x}\\
						&=	0+(\alpha-1)\Gamma(\alpha-1)\\
						&=	(\alpha-1)\Gamma(\alpha-1)
	\end{split}
\end{equation}
Therefore,
\begin{equation}\label{eq:gammafunction:closedform}
	\Gamma(\alpha)	=	(\alpha-1)\Gamma(\alpha-1)\textnormal{ for }\alpha\in(1,\infty)
\end{equation}
Furthermore, $\Gamma(1)$ is defined separately as being:
\vspace*{-12pt}
\begin{equation}
	\Gamma(1) = \int^\infty_0 e^{-x}\diff{x} = \left[-e^{-x}\right]^\infty_0 = 1
\end{equation}

%======================================================================================================================
%======================================================================================================================
%=================================================================================================================

\pagebreak
\subsubsection{Special Case of $\Gamma(\sfrac{1}{2})$}
An extremely interesting case of the gamma function is $\Gamma(\sfrac{1}{2})$. In fact, this result allows the
computation of fractional values of $\alpha$ such that $\alpha\in\mathbb{Q}$. This special case is as follows:
\begin{claim}\label{claim:gammafunction:onehalf}
	\begin{equation*}
		\Gamma(\sfrac{1}{2})=\sqrt{\pi}
	\end{equation*}
\end{claim}

\paragraph{Initial Setup}
First, to prove Claim \ref{claim:gammafunction:onehalf}, let's start by setting up some equations in preparation for the
proof.
\begin{equation}\label{eq:gammafunction:i}
	I = \int^\infty_0 e^{-x^2}\diff{x}
\end{equation}
From this result, one obtains that:
\begin{equation}\label{eq:gammafunction:isquared}
	\begin{split}
		I^2	&=	\left(\int^\infty_0 e^{-x^2}\diff{x}\right)\left(\int^\infty_0 e^{-y^2}\diff{y}\right)\\
			&=	\int^\infty_0\int^\infty_0 e^{-(x^2+y^2)}\diff{x}\diff{y}
	\end{split}
\end{equation}
To solve this integral easily, one may use the following change of variables:
\begin{align*}
	x=r\cos\theta	&&	y=r\sin\theta
\end{align*}
This change of variable requires the Jacobian of the transformation to be computed:
\begin{equation}\label{eq:gammafunction:jacobian}
	\J = \det\begin{bmatrix}
		\pderiv{x}{r}	&	\pderiv{x}{\theta}\\[6pt]
		\pderiv{y}{r}	&	\pderiv{y}{\theta}
		\end{bmatrix} = \det\begin{bmatrix}
			\cos\theta	&	-r\sin\theta\\
			\sin\theta	&	r\cos\theta
		\end{bmatrix} = r(\cos^2\theta + \sin^2\theta) = r
\end{equation}
Using the result obtained in \autoref{eq:gammafunction:jacobian}, the integral in \autoref{eq:gammafunction:isquared}
may be computed using polar coordinates:
\begin{equation}
	\begin{split}
		I^2	=	\int^\infty_0\int^\infty_0 e^{-(x^2+y^2)}\diff{x}\diff{y}	&=	\int^\frac{\pi}{2}_0\int^\infty_0 re^{-r^2}\diff{r}\diff{\theta}\\
																			&=	\frac{\pi}{2}\int^\infty_0 re^{-r^2}\diff{r}\\
																			&=	\frac{\pi}{2}\left[-\frac{e^{-r^2}}{2}\right]^\infty_0\\
																			&=	\frac{\pi}{2}\frac{1}{2}\\
																			&=	\frac{\pi}{4}
	\end{split}
\end{equation}
It then follows that \autoref{eq:gammafunction:i} may be solved:
\begin{equation}
	I = \sqrt{I^2} = \int^\infty_0 e^{-x^2}\diff{x} = \sqrt{\frac{\pi}{4}} = \frac{\sqrt{\pi}}{2}
\end{equation}
Accordingly, Corollaries \autoref{corollary:gammafunction:sqrtpi} and \autoref{corollary:gammafunction:sqrt2pi}
logically follow:
\begin{corollary}\label{corollary:gammafunction:sqrtpi}
	\begin{equation}
		\int^\infty_{-\infty} e^{-x^2}\diff{x} = \sqrt{\pi}
	\end{equation}
\end{corollary}
\begin{corollary}\label{corollary:gammafunction:sqrt2pi}
	\begin{equation}
		\int^\infty_{-\infty} e^{\sfrac{-x^2}{2}}\diff{x} = \sqrt{2\pi}
	\end{equation}
\end{corollary}
\noindent Let's prove these corollaries:\\[-36pt]
\begin{proof}
	Consider this substitution of variables:\\[-36pt]
	\begin{align*}
		z = \sqrt{2}x	&&	\diff{z} = \sqrt{2}\diff{x}
	\end{align*}\\[-36pt]
	Using this substitution, Corollaries \autoref{corollary:gammafunction:sqrtpi} and
	\autoref{corollary:gammafunction:sqrt2pi} may be proven:\\[-18pt]
	\begin{equation}
		\int^\infty_{-\infty} e^{-x^2}\diff{x} = \frac{1}{\sqrt{2}}\int^\infty_{-\infty} e^{\sfrac{-z^2}{2}}\diff{z} = \sqrt{\pi}\\
	\end{equation}
	\vspace*{-30pt}
	\begin{equation}
		\int^\infty_{-\infty} e^{\sfrac{-z^2}{2}}\diff{z} = \sqrt{2}\sqrt{\pi} = \sqrt{2\pi}
	\end{equation}
\end{proof}
\paragraph{The Proof}
The proof for Claim \ref{claim:gammafunction:onehalf} may now be articulated:\\[-36pt]
\begin{proof}
	Consider, once again, this other substitution of variables:\\[-36pt]
	\begin{align*}
		x = \sqrt{z}	&&	\diff{x} = \frac{1}{2\sqrt{z}}\diff{z} = \frac{z^{-\sfrac{1}{2}}}{2}\diff{z}
	\end{align*}\\[-36pt]
	Using it, the following integral may be solved:
	\begin{equation}
		\begin{split}
			\int^\infty_0 e^{-x^2}\diff{x}	&=	\frac{1}{2}\int^\infty_0 z^{-\sfrac{1}{2}}e^{-z}\diff{z}\\
											&=	\frac{1}{2}\Gamma(\sfrac{1}{2})\\
											&=	\frac{\sqrt{\pi}}{2}
		\end{split}
	\end{equation}
	Conclusively, Claim \ref{claim:gammafunction:onehalf} is proven:
	\begin{equation}
		\Gamma(\sfrac{1}{2}) = \sqrt{\pi}
	\end{equation}
\end{proof}


%======================================================================================================================
%======================================================================================================================
%======================================================================================================================

\pagebreak
\subsubsection{Negative Values of $\alpha$}
As was mentioned earlier, by abusing the identity shown in \autoref{eq:gammafunction:closedform}, one may extend the
gamma function to negative non-integer values of $\alpha$ such that $\alpha\in\mathbb{R}\backslash\mathbb{Z}^-$. For
example, to find $\Gamma(-\sfrac{9}{2})$, one may compute:
\vspace*{-12pt}
\begin{equation}
	\begin{split}
		\Gamma(\sfrac{1}{2})	&=	(-\sfrac{1}{2})\Gamma(-\sfrac{1}{2})\\
								&=	(-\sfrac{1}{2})(-\sfrac{3}{2})\Gamma(-\sfrac{3}{2})\\
								&=	(\sfrac{3}{4})(-\sfrac{5}{2})\Gamma(-\sfrac{5}{2})\\
								&=	(-\sfrac{15}{8})(-\sfrac{7}{2})\Gamma(-\sfrac{7}{2})\\
								&=	(\sfrac{105}{16})(-\sfrac{9}{2})\Gamma(-\sfrac{9}{2})\\
								&=	(-\sfrac{945}{32})\Gamma(-\sfrac{9}{2})\\
	\end{split}
\end{equation}
\begin{equation}
	\therefore\Gamma(-\sfrac{9}{2})=-\frac{32}{945}\Gamma(\sfrac{1}{2})=-\frac{32}{945}\sqrt{\pi}
\end{equation}
Hence, to calculate the value of a gamma function with a negative argument, one might modify a bit the result obtained
in \autoref{eq:gammafunction:closedform} such that:
\vspace*{-12pt}
\begin{equation}
	\Gamma(\alpha)=\frac{\Gamma(\alpha+1)}{\alpha}\text{ for }\alpha\in\mathbb{R}^-\backslash\mathbb{Z}
\end{equation}\\[-36pt]
for example, take the same example of $\Gamma(-\sfrac{9}{2})$\\[-12pt]
\begin{equation}
	\begin{split}
		\Gamma(-\sfrac{9}{2})	&=	\frac{\Gamma(-\sfrac{7}{2})}{(-\sfrac{9}{2})}\\
								&=	\frac{\Gamma(-\sfrac{5}{2})}{(-\sfrac{9}{2})(-\sfrac{7}{2})}\\
								&=	\frac{\Gamma(-\sfrac{3}{2})}{(\sfrac{63}{4})(-\sfrac{5}{2})}\\
								&=	\frac{\Gamma(-\sfrac{1}{2})}{(-\sfrac{315}{8})(-\sfrac{3}{2})}\\
								&=	\frac{\Gamma(\sfrac{1}{2})}{(\sfrac{945}{16})(-\sfrac{1}{2})}\\
								&=	\frac{\Gamma(\sfrac{1}{2})}{(-\sfrac{945}{32})}
		\end{split}
\end{equation}
\begin{equation}
	\therefore\Gamma(-\sfrac{9}{2})=-\frac{32}{945}\Gamma(\sfrac{1}{2})=-\frac{32}{945}\sqrt{\pi}
\end{equation}

%======================================================================================================================
%======================================================================================================================
%======================================================================================================================

\pagebreak
\subsection{Relationship to Factorial Notation}
Using a proof by induction, gamma functions can be shown to be related to the factorial notation.
\begin{proof}
	Let us prove \autoref{eq:gammafunction:factorial}, that is, that:\\[-18pt]
	\begin{equation}
		\Gamma(n) = \int^\infty_0 x^{n-1}e^{-x}\diff{x} = (n-1)!\text{ for }n\in\mathbb{N}
	\end{equation}\\[-30pt]
	For $n=1$:\\[-18pt]
	\begin{equation}
		\begin{split}
			\Gamma(1)	&=	\int^\infty_0 x^{1-1}e^{-x}\diff{x}\\
						&=	\int^\infty_0 e^{-x}\diff{x}\\
						&=	1\\
						&=	(1-1)!\\
						&=	0!
		\end{split}
	\end{equation}\\[-24pt]
	Assume that the equation holds for $n=k$ such that:\\[-18pt]
	\begin{equation}
		\Gamma(k) = (k-1)!
	\end{equation}\\[-30pt]
	Then, let us prove that the equation is valid for $n=k+1$ using integration by parts:\\[-12pt]
	\begin{empheq}[box=\widefbox]{align*}
		u=x^{k};					&&	\diff{v}=e^{-x}\diff{x}\\
		\diff{u}=kx^{k-1}\diff{x};	&&	v=-e^{-x}
	\end{empheq}
	\begin{equation}
		\begin{split}
			\Gamma(k+1)	&=	\int^\infty_0 x^{(k+1)-1}e^{-x}\diff{x}\\
						&=	\left[-x^k e^{-x}\right]^\infty_0 + k\int^\infty_0 x^{k-1}e^{-x}\diff{x}\\
						&=	0 +k\Gamma(k)\\
						&=	k(k-1)!\\
						&=	((k+1)-1)!\\
						&=	k!
		\end{split}
	\end{equation}
	Therefore, \autoref{eq:gammafunction:factorial} is proven by mathematical induction.
\end{proof}

%======================================================================================================================
%======================================================================================================================
%======================================================================================================================

\pagebreak
\subsection{Plotting the Gamma Function}
Using everything that was computed previously, the graph of $\Gamma(\alpha)$ may be plotted. It includes both positive,
negative and fractional values of $\alpha$. Notice how the asymptotes of the graph are found at every negative integer
value; the function, as was mentioned previously, is not defined for negative integers.
\figurecenter{1}{gamma_function.png}{Plot of the function $\Gamma(\alpha)$ in red and the asymptotes of the function in dotted blue lines.}{fig:graph:gammafunction}

%======================================================================================================================
%======================================================================================================================
%======================================================================================================================

\pagebreak
\section{Special Cases of the Gamma Distribution \& Related Proofs}
The gamma distribution has many common parameterizations. Hence, in order to make working with them easier, they were
each given a specific name such as the chi-square distribution, the exponential distribution and more. In this section,
we will present some of these special cases and prove their relationship.

\subsection{The Exponential Distribution}\label{subsec:specialcases:exponential}
The gamma distribution can model the elapsed time between random and independent events. However, what should one do
when one wants to strictly model the time to wait before a single new event?

If a unique event is modelled, it follows that the shape parameter is limited to $\alpha = 1$. In such a case, the gamma
distribution is said to follow an exponential distribution: more specifically, if $X\sim\G\text{amma}(1, \lambda)$, then
it can also be said that $X\sim\E\text{xp}(\lambda)$. The continuous random variable $X$ thus follows an exponential
distribution with rate parameter $\lambda$, where the rate parameter represents how quickly events occur. More
concretely:
\begin{equation}
	X\sim\G\text{amma}(1, \lambda)\equiv\E\text{xp}(\lambda)
\end{equation}
Indeed, using a simple substitution of the variables, one gets that:
\begin{equation}\label{eq:relation:exp}
	\begin{split}
		f(x)	&=	\frac{\lambda^\alpha x^{\alpha-1}}{\Gamma(\alpha)}e^{-\lambda x}\\
				&=	\frac{\lambda^1 x^{0}}{\Gamma(1)}e^{-\lambda x}\\
				&=	\lambda e^{-\lambda x}
	\end{split}
\end{equation}
\begin{equation}
	\therefore X\sim\G\text{amma}(1, \lambda)\equiv X\sim\E\text{xp}(\lambda)
\end{equation}
Accordingly, it can be demonstrated that the sum of exponential random variables is a gamma random variable. That is, if
$X_1, X_2, \ldots, X_n$ are i.i.d random variables following $\E\text{xp}(\lambda)$, then:
\begin{equation}
	Y=\sum^k_{i=1}\left[X_i\right]\sim\G\text{amma}(k, \lambda)
\end{equation}

\subsubsection{Discovery of the Gamma Probability Density Function}
Here, we propose that the PDF of the gamma distribution was found by adding multiple i.i.d random variables following an
exponential distribution $\E\text{xp}(\lambda)$. Here is a proof by induction demonstrating this supposition
\cite{ayiendaGammaRelatedDistributions2013}:
\begin{proof}
	First, let us find the sum of two exponential random variables, which we will denote $S_2$:
	\begin{equation}
		\begin{split}
			f_{S_2}	&=	f_{X_1 + X_2} (t)\\
					&=	\int^t_0 \lambda e^{-\lambda(t-s)}\lambda e^{-\lambda s}\diff{s}\\
					&=	\lambda^2 e^{-\lambda t}\int^t_0 1\diff{s}\\
					&=	\lambda^2 te^{-\lambda t}\\
					&=	\frac{\lambda^2 t^{2-1} e^{-\lambda t}}{(2-1)!}
		\end{split}
	\end{equation}
	With this result, we can now determine the probability distribution function of the sum of three, as well as four
	exponential random variables. Let's start with the sum of three variables:
	\begin{equation}
		\begin{split}
			f_{X_1 + X_2 + X_3}(t)	&=	f_{S_2 + X_3}(t)\\
			 						&=	\int^\infty_{-\infty} f_{S_2}(t-s)f_{X_3}(s)\diff{s}\\
			 						&=	\int^\infty_{-\infty} f_{X_3}(t-s)f_{S_2}(s)\diff{s}\\
			 						&=	\int^t_0 \lambda e^{-\lambda(t-s)}\lambda^2 s e^{-\lambda s}\diff{s}\\
									&=	\lambda^3 e^{-\lambda t} \int^t_0 s\diff{s}\\
									&=	\frac{\lambda^3 t^2 e^{-\lambda t}}{2!}\\
									&=	\frac{\lambda^3 t^{3-1} e^{-\lambda t}}{(3-1)!}
		\end{split}
	\end{equation}
	Then, let's sum four variables:
	\begin{equation}
		\begin{split}
			f_{X_1+X_2+X_3+X_4}(t)	&=	f_{S_3+X_4}(t)\\
									&=	\int^\infty_{-\infty} f_{S_3}(t-s)f_{X_4}(s)\diff{s}\\
									&=	\int^\infty_{-\infty} f_{X_4}(t-s)f_{S_3}(s)\diff{s}\\
									&=	\int^t_0 \lambda e^{-\lambda(t-s)}\frac{\lambda^3 s^2 e^{-\lambda s}}{2!}\diff{s}\\
									&=	\frac{\lambda^4 e^{-\lambda t}}{2!}\int^t_0 s^2\diff{s}\\
									&=	\frac{\lambda^4 t^3 e^{-\lambda t}}{3!}\\
									&=	\frac{\lambda^4 t^{4-1} e^{-\lambda t}}{(4-1)!}\\
		\end{split}
	\end{equation}
	A pattern can clearly be seen emerging from those results. Hence, we make the claim that:
	\begin{equation}
		\begin{split}
			f_{S_n}(t) = f_{\sum^n_{i=1}X_i}(t) = \frac{\lambda^n t^{n-1} e^{-\lambda t}}{(n-1)!}\\
		\end{split}
	\end{equation}
	To prove this claim, let us assume that:
	\begin{equation}
		\begin{split}
			f_{S_{n-1}}(t)	&=	f_{\sum^{n-1}_{i=1}X_i} (t)\\
							&=	\frac{\lambda^{n-1} t^{n-2} e^{-\lambda t}}{(n-2)!}
		\end{split}
	\end{equation}
	By convolution, the probability distribution function of an infinite sum of exponential random variables can now be
	found.
	\begin{equation}
		\begin{split}
			f_{S_n}(t)	&=	f_{\sum^n_{i=1}X_i}(t)\\
						&=	f_{\sum^{n-1}_{i=1}[X_i] + X_n }(t)\\
						&=	f_{S_{n-1} + X_n }(t)\\
						&=	\int^\infty_{-\infty} f_{S_{n-1}}(t-s)f_{X_n}(s)\diff{s}\\
						&=	\int^\infty_{-\infty} f_{X_n}(t-s)f_{S_{n-1}}(s)\diff{s}\\
						&=	\int^t_0 \lambda e^{-\lambda(t-s)}\frac{\lambda^{n-1} s^{n-2} e^{-\lambda s}}{(n-2)!}\diff{s}\\
						&=	\frac{\lambda^n e^{-\lambda t}}{(n-2)!} \int^t_0 s^{n-2}\diff{s}\\
						&=	\frac{\lambda^n t^{n-1} e^{-\lambda t}}{(n-1)!}\\
						&=	\frac{\lambda^n t^{n-1} e^{-\lambda t}}{\Gamma(n)}\\
		\end{split}
	\end{equation}
	This corresponds, obviously, to the probability distribution function of a gamma distribution. Therefore:
	\begin{equation}
		S_n \sim Gamma(n, \lambda)
	\end{equation}
	Hence, we get a density function whose parameter $n$ can correspond to all real values instead of being limited to
	natural numbers.
\end{proof}

%======================================================================================================================
%======================================================================================================================
%======================================================================================================================

\pagebreak
\subsubsection{Sum of Exponential Random Variables}
It is interesting to know that the gamma distribution is infinitely divisible. That is, any random variable following a
gamma distribution $\G\text{amma}(\alpha, \lambda)$ may be expressed as a sum of $n$ other random variables following a
gamma distribution as long as their shape parameters add up to $\alpha$ and their rate parameter is $\lambda$.
\paragraph{Using Convolution}
Using the notion of convolution, then
\begin{equation}
	\begin{split}
		f(z)	&=	\int_{0}^{z} f_1 (z-y) f_2 (y) \diff{y}\\
				&=	\int_{0}^{z}\frac{\lambda^{\alpha_1} e^{-\lambda(z-y)} (z-y)^{\alpha_1 -1}}{\Gamma(\alpha_1)} \cdot \frac{\lambda^{\alpha_2} e^{-\lambda y}y^{\alpha_2 -1}}{\Gamma(\alpha_2)}\diff{y}\\
				&=	\frac{\lambda^{\alpha_1+\alpha_2}}{\Gamma(\alpha_1) \Gamma(\alpha_2)}\int_{0}^{z}e^{-\lambda z}(z-y)^{\alpha_1 -1}y^{\alpha_2 -1}\diff{y}\\
	\end{split}
\end{equation}

\begin{empheq}[box=\widefbox]{align*}
	y = zt;	&&	\diff{y} = z\diff{t};	&&	t = \frac{y}{z}
\end{empheq}

Therefore,
\begin{equation}
	\begin{split}
		 f(z)	&=	\frac{\lambda^{\alpha_1+\alpha_2}}{\Gamma(\alpha_1)\Gamma(\alpha_2)}\int_{0}^{1}e^{-\lambda z}(z-zt)^{\alpha_1 -1}(zt)^{\alpha_2 -1}z\diff{t}\\
				&=	\frac{\lambda^{\alpha_1+\alpha_2}}{\Gamma(\alpha_1) \Gamma(\alpha_2)}\int_{0}^{1}z^{\alpha_1 + \alpha_2 -1}e^{-\lambda z}(1-t)^{\alpha_1 -1} t^{\alpha_2 -1}\diff{t}\\
				&=	\frac{\lambda^{\alpha_1+\alpha_2}}{\Gamma(\alpha_1) \Gamma(\alpha_2)} z^{\alpha_1 + \alpha_2 -1}e^{-\lambda z} \int_{0}^{1}(1-t)^{\alpha_1 -1} t^{\alpha_2 -1}\diff{t}\\
				&=	\frac{\lambda^{\alpha_1+\alpha_2}}{\Gamma(\alpha_1) \Gamma(\alpha_2)} z^{\alpha_1 + \alpha_2 -1}e^{-\lambda z} B(\alpha_2,\alpha_1)\\
	\end{split}
\end{equation}
Here, let us now use an alternative definition of the beta distribution to obtain our final result:
\begin{equation}
	\begin{split}
	f(z)	&=	\frac{\lambda^{\alpha_1+\alpha_2} }{\Gamma(\alpha_1) \Gamma(\alpha_2)} z^{\alpha_1 + \alpha_2 -1}e^{-\lambda z} \cdot \frac{\Gamma(\alpha_1) \Gamma(\alpha_2)}{\Gamma(\alpha_1+\alpha_2)}\\
			&=	\frac{\lambda^{\alpha_1+\alpha_2}}{\Gamma(\alpha_1+\alpha_2)} z^{\alpha_1 + \alpha_2 -1}e^{-\lambda z} \text{ for } z>0, \alpha_1 + \alpha_2 -1 >0\\
	\end{split}
\end{equation}
Since f(z) is in fact a probability density function, then:
\begin{equation}
	\begin{split}
		1	&=	\int_{0}^{\infty} \frac{\lambda^{\alpha_1+\alpha_2}}{\Gamma(\alpha_1) \Gamma(\alpha_2)} z^{\alpha_1 + \alpha_2 -1}e^{-\lambda z} B(\alpha_2,\alpha_1)\diff{z}\\
			&=	\frac{\lambda^{\alpha_1+\alpha_2}}{\Gamma(\alpha_1) \Gamma(\alpha_2)} B(\alpha_2,\alpha_1) \int_{0}^{\infty}e^{-\lambda z}z^{\alpha_1 + \alpha_2 -1}\diff{z}\\
	\end{split}
\end{equation}
Finally, using the definition of the gamma distribution, we arrive to:
\begin{equation}
	\begin{split}
		1	&=	\frac{\lambda^{\alpha_1+\alpha_2}}{\Gamma(\alpha_1) \Gamma(\alpha_2)} B(\alpha_2,\alpha_1) \frac{\Gamma({\alpha_1+\alpha_2})}{\lambda^{\alpha_1+\alpha_2}}\\
			&=	\frac{B(\alpha_2,\alpha_1) \Gamma({\alpha_1+\alpha_2}) }{\Gamma(\alpha_1) \Gamma(\alpha_2)}\\
	\end{split}
\end{equation}
Thus, as a by-product, we have found that:
\begin{equation}\label{eq:specialcases:beta:beta}
	\begin{split}
		B(\alpha_2,\alpha_1) = \frac{\Gamma(\alpha_1) \Gamma(\alpha_2)}{\Gamma({\alpha_1+\alpha_2})}
	\end{split}
\end{equation}
thereby confirming our initial assumption about the alternative definition of the beta distribution.

%======================================================================================================================
%======================================================================================================================
%======================================================================================================================

\pagebreak
\paragraph{Using a Change of Variables}
We will now prove this same result using a change of variables technique. First, let $Z = X + Y$. We will now find the
distribution function of $Z$:
\begin{equation}
	\begin{split}
		F_{X+Y}(z)	&=	P(Z \leq z)\\
					&=	P(X+Y \leq z)\text{ for } x, y\in(0, \infty)\\
					&=	P(X \leq z - y)\\
					&=	\int_{0}^{\infty} \int_{0}^{z-y} f_X (x) f_Y (y)\diff{x}\diff{y}\\
					&=	\int_{0}^{\infty} F_x (z-y) f_Y (y)\diff{y}\\
	\end{split}
\end{equation}
The associated probability density function can now be found.
\begin{equation}
	\begin{split}
		f_{x+y}(z)	&=	\int_{0}^{\infty} f_X(z-y)f_Y (y)\diff{y} \text{ for } z\in(0, \infty)\\
					&=	\int_{0}^{\infty} \int_{0}^{z-y} f_x(x)f_y(y)\diff{x}\diff{y}\\
					&=	\int_{0}^{\infty} \frac{\lambda^{\alpha_1}}{\Gamma(\alpha_1)} (z-y)^{\alpha_1 - 1} e^{-\lambda(z-y)} \frac{\lambda^{\alpha_2}}{\Gamma(\alpha_2)} y^{\alpha_2 - 1} e^{-\lambda y}\diff{y}\\
					&=	\frac{\lambda^{\alpha_1} \lambda^{\alpha_2}}{\Gamma(\alpha_1) \Gamma(\alpha_2)} \int_{0}^{\infty} (z-y)^{\alpha_1 - 1} e^{-\lambda(z-y)} y^{\alpha_2 - 1} e^{-\lambda y}\diff{y}\\
					&=	\frac{\lambda^{\alpha_1} \lambda^{\alpha_2}}{\Gamma(\alpha_1) \Gamma(\alpha_2)} \int_{0}^{\infty} (z-y)^{\alpha_1 - 1} e^{\lambda(-z+y-y)} y^{\alpha_2 - 1}\diff{y}\\
	\end{split}
\end{equation}

\begin{empheq}[box=\widefbox]{align*}
	y = zu;	&&	\diff{y} = z\diff{u}
\end{empheq}
Let us now make a simple substitution so that the final answer can be revealed.
\begin{equation}
	\begin{split}
		f_{X+Y}(z)	&=	\frac{\lambda^{\alpha_1 + \alpha_2}}{\Gamma(\alpha_1) \Gamma(\alpha_2)} e^{-\lambda z} \int_{0}^{\infty} (z-zu)^{\alpha_1 - 1} (zu)^{\alpha_2 - 1}z \diff{u}\\
					&=	\frac{\lambda^{\alpha_1 + \alpha_2}}{\Gamma(\alpha_1) \Gamma(\alpha_2)} e^{-\lambda z} \int_{0}^{\infty} z^{\alpha_1 + \alpha_2} (1-u)^{\alpha_1 - 1} u^{\alpha_2 - 1}\diff{u}\\
					&=	\frac{ \lambda^{\alpha_1 + \alpha_2}}{\Gamma(\alpha_1) \Gamma(\alpha_2)} e^{-\lambda z} z^{\alpha_1 + \alpha_2 - 1} \int_{0}^{\infty} u^{\alpha_2 - 1} (1-u)^{\alpha_1 - 1}\diff{u}\\
	\end{split}
\end{equation}
We will note here that the integral part of this equation corresponds to the Beta distribution function. Hence, we get:\vspace*{-18pt}
\begin{equation}
		f_{X+Y} (z) = \frac{\lambda^{\alpha_1 + \alpha_2}}{\Gamma(\alpha_1) \Gamma(\alpha_2)} e^{-\lambda z} z^{\alpha_1 + \alpha_2 - 1} B(\alpha_1, \alpha_2)
\end{equation}\\[-30pt]
Using an alternative definition of the beta distribution, which was proven in the last demonstration, we arrive to our final answer:\vspace*{-18pt}
\begin{equation}
		f_{X+Y}(z) = \frac{\lambda^{\alpha_1 + \alpha_2}}{\Gamma(\alpha_1 + \alpha_2)} e^{-\lambda z} z^{(\alpha_1 + \alpha_2) - 1}
\end{equation}\\[-30pt]
This corresponds, most clearly, to the probability density function of a gamma distribution.\vspace*{-18pt}
\begin{equation}
	Z \sim \G\text{amma}({\alpha_1 + \alpha_2, \lambda})
\end{equation}\\[-30pt]
Here is a graph showing this result:
\figurecenter{1}{Simulation_sum.png}{Comparison of the histogram of a sum of gamma random variables with the probability density function of a gamma distribution.}{fig:simulation:sum}


%======================================================================================================================
%======================================================================================================================
%======================================================================================================================

\pagebreak
\subsection{The Beta Distribution}\label{subsec:specialcases:beta}
A random variable $X$ is said to follow a beta distribution $\text{Beta}(\alpha, \alpha)$ with two parameters $\alpha$
and $\beta$ if its probability distribution function is defined by:
\begin{equation}\label{eq:specialcases:beta:pdf}
	f(x) = \frac{x^{\alpha-1} (1-x)^{\beta-1}}{B(\alpha, \beta)}\text{ for } 0\leq x\leq 1\text{ and }\alpha, \beta > 0
\end{equation}
where, from \autoref{eq:specialcases:beta:beta}:
\begin{equation}
	B(\alpha, \beta) = \frac{\Gamma(\alpha)\Gamma(\beta)}{\Gamma(\alpha+\beta)}
\end{equation}
 Here is a graph of the probability density function:
\figurecenter{1}{Beta_PDF.png}{Plot of the probability distribution function of a beta distribution for various parameterizations.}{fig:specialcases:beta:pdf}

%======================================================================================================================
%======================================================================================================================
%======================================================================================================================

\pagebreak
When looking at the numerator of the beta distribution function, similarities with the distribution function of the
binomial distribution for independently distributed random events are evident.
\begin{equation}
	p(x) = P(X=x) =\binom{n}{x}p^x(1-p)^{n-x}\text{ for }x\in\mathbb{N}\cup\{0\}
\end{equation}
Indeed, the two distribution functions are analogous to a certain extent, for while the binomial distribution models the
number of successes, the beta distribution models the probability of success. Since $\alpha-1$ represents the number
of successes and $\beta-1$ represents the number of failures, modifying those parameters changes the shape of the
probability distribution to model varying probabilities. As $\alpha$ increases (more successful events), the bulk of the
probability distribution will shift towards the right, whereas an increase in $\beta$ moves the distribution towards the
left (more failures). Finally, the distribution will narrow if both $\alpha$ and $\beta$ increase, for this increases the
certainty regarding possible outcomes \cite{kimBetaDistributionIntuition2020, disorboBetaGamma}.

%======================================================================================================================
%======================================================================================================================
%======================================================================================================================

\pagebreak
\subsubsection{Quotient of Gamma Random Variables}\label{subsubsec:specialcases:beta:quotient}
Let's show how this distribution is related to the gamma distribution. Indeed, for $X\sim\G\text{amma}(\alpha, \lambda)$
and $Y\sim\G\text{amma}(\beta, \lambda)$, then
$\frac{X}{X+Y}\sim\G\text{amma}(\alpha+\beta, \lambda)\equiv\text{Beta}(\alpha, \beta)$. Let's prove it using two
different techniques \cite{ayiendaGammaRelatedDistributions2013}:

\paragraph{Using a Jacobian}
\begin{proof}
	First, let:
	\begin{empheq}[box=\widefbox]{align*}
		Y_1 = X_1 + X_2;	&&	Y_2 = \frac{X_1}{X_1+X_2}
	\end{empheq}
	Therefore:
	\begin{empheq}[box=\widefbox]{align*}
		x_1 = y_1y_2;	&&	x_2 = y_1(1-y_2)
	\end{empheq}
	\begin{equation}
		\J = \det\begin{bmatrix}
			\pderiv{x_1}{y_1}	&	\pderiv{x_1}{y_1}\\[6pt]
			\pderiv{x_2}{y_2}	&	\pderiv{x_2}{y_2}
			\end{bmatrix} = \det\begin{bmatrix}
				y_2		&	y_1\\
				1-y_2	&	-y_1
			\end{bmatrix} = -y_1
	\end{equation}
	\begin{equation}
		\begin{split}
			f(y_1, y_2)	&=	f(x_1, x_2)\cdot\J\\
						&=	\frac{\lambda^{\alpha_1+\alpha_2}}{\Gamma(\alpha_1)\Gamma(\alpha_2)}x_1^{\alpha_1-1}x_2^{\alpha_2-1}e^{-\lambda(x_1+x_2)}\text{ for }x_1, x_2>0\\
						&=	\frac{\lambda^{\alpha_1+\alpha_2}}{\Gamma(\alpha_1)\Gamma(\alpha_2)}(y_1y_2)^{\alpha_1-1}(y_1(1-y_2))^{\alpha_2-1}e^{-\lambda y_1}y_1\\
						&=	\frac{\lambda^{\alpha_1+\alpha_2}}{\Gamma(\alpha_1)\Gamma(\alpha_2)}y_1^{\alpha_1+\alpha_2-1}y_2^{\alpha_2-1}(1-y_2)^{\alpha_2-1}e^{-\lambda y_1}
		\end{split}
	\end{equation}
	Hence:
	\begin{equation}
		\begin{split}
			f(y_2)	&=	\int^\infty_0 f(y_1, y_2)\diff{y_1}\\
					&=	\int^\infty_0 \frac{\lambda^{\alpha_1+\alpha_2}}{\Gamma(\alpha_1)\Gamma(\alpha_2)}y_1^{\alpha_1+\alpha_2-1}y_2^{\alpha_2-1}(1-y_2)^{\alpha_2-1}e^{-\lambda y_1}\diff{y_1}
		\end{split}
	\end{equation}
	\pagebreak

	\noindent The beta distribution may now be obtained through the following calculations:
	\begin{empheq}[box=\widefbox]{align*}
		y = \lambda y_1 \Rightarrow y_1 = \frac{y}{\lambda}
	\end{empheq}
	\begin{equation}
		\begin{split}
			\frac{1}{\lambda}\diff{y}	&=	\diff{y_1}\\
										&=	\frac{\lambda^{\alpha_1+\alpha_2}y_2^{\alpha_2-1}(1-y_2)^{\alpha_2-1}}{\Gamma(\alpha_1)\Gamma(\alpha_2)} \int^\infty_0 \frac{e^{-y}}{\lambda}\left(\frac{y}{\lambda}\right)^{\alpha_1+\alpha_2-1}\diff{y}\\
										&=	\frac{\lambda^{\alpha_1+\alpha_2}y_2^{\alpha_2-1}(1-y_2)^{\alpha_2-1}}{\lambda^{\alpha_1+\alpha_2}\Gamma(\alpha_1)\Gamma(\alpha_2)} \int^\infty_0 e^{-y}y^{\alpha_1+\alpha_2-1}\diff{y}\\
										&=	\frac{y_2^{\alpha_2-1}(1-y_2)^{\alpha_2-1}\Gamma(\alpha_1+\alpha_2)}{\Gamma(\alpha_1)\Gamma(\alpha_2)}\\
										&=	\frac{y_2^{\alpha_2-1}(1-y_2)^{\alpha_2-1}}{B(\alpha_1, \alpha_2)}\text{ for }0\leq y\leq 1\text{ and }\alpha_1, \alpha_2 > 0
		\end{split}
	\end{equation}
	This last result is exactly the equation of the beta distribution presented in \autoref{eq:specialcases:beta:pdf}, but with
	different variables.
\end{proof}
%======================================================================================================================
%======================================================================================================================
%======================================================================================================================

\pagebreak
\paragraph{Using a Change of Variables}
\begin{proof}
	Alternatively, using a change of variables, consider the following random variable:\vspace*{-12pt}
	\begin{equation}
		Z = \frac{X_1}{X_1+X_2}\\
	\end{equation}\\[-38pt]
	then:\vspace*{-12pt}
	\begin{equation}
		\begin{split}
			P(Z\le z)	&=	P(\frac{X_1}{X_1+X_2}\le z)\\
						&=	P(X_1\le z (x_1+ x_2))\\
						&=	P((1-z)X_1\le zx_2)\\
						&=	P(X_1\le {\frac{zx_2}{1-z}}) \text{ for } X_2 \in (0,\infty)\\
						&=	\int_{0}^{\infty}\int_{0}^{\frac{zx_2}{1-z}} f(x_1)\diff{x_1} f(x_2)\diff{x_2}\\
						&=	\int_{0}^{\infty}F_{X_1}(\frac{z x_2}{1-z})f(x_2)\diff{x_2}\\
			h(z)		&=	\int_{0}^{\infty}\frac{x_2}{(1-z)^2}f_{x_1}(\frac{zx_2}{1-z})f(x_2)\diff{x_2}\\
						&=	\frac{\lambda^{\alpha_1+\alpha_2} z^{\alpha_1 - 1}}{\Gamma (\alpha _1) \Gamma (\alpha _2) (1-z)^{\alpha_1 -+1}} \int_{0}^{\infty} x_2^{\alpha_1 + \alpha_2 - 1} e^{-\lambda \frac{x_2}{(1-z)}}\diff{x_2}\\
						&=	\frac{\lambda^{\alpha_1+\alpha_2} z^{\alpha_1 - 1}}{\Gamma (\alpha _1) \Gamma (\alpha _2) (1-z)^{\alpha_1 +1}} \int_{0}^{\infty} x_2^{\alpha_1 + \alpha_2 - 1} e^{-\lambda \frac{x_2}{(1-z)}}\diff{x_2}\\
		\end{split}
	\end{equation}
	\begin{empheq}[box=\widefbox]{align*}
		u = \frac{\lambda x_2}{1-z};	&&	x_2 = \frac{u(1-z)}{\lambda};	&&	\diff{x_2} = \frac{(1-z)}{\lambda}\diff{u}
	\end{empheq}
	\begin{equation}
		\begin{split}
			h(z)	&=	\frac{\lambda^{\alpha_1+\alpha_2} z^{\alpha_1 - 1}}{\Gamma (\alpha _1) \Gamma (\alpha _2) (1-z)^{\alpha_1 +1}} \int_{0}^{\infty} (\frac{u (1-z)}{\lambda})^{\alpha_1 + \alpha_2 -1} e^{-u} (\frac{1-z}{\lambda})\diff{u}\\
					&=	\frac{\lambda^{\alpha_1+\alpha_2} (1-z)^{\alpha_1 + \alpha_1}}{\lambda^{\alpha_1+\alpha_2} \Gamma (\alpha _1) \Gamma (\alpha _2) (1-z)^{\alpha_1 +1}} \int_{0}^{\infty} u^{\alpha_1 + \alpha_2 -1} e^{-u}\diff{u}\\
					&=	\frac{z^{\alpha_1 -1} (1-z)^{\alpha_2 -1}}{\Gamma (\alpha _1) \Gamma (\alpha _2)} \int_{0}^{\infty} u^{\alpha_1 + \alpha_2 -1} e^{-u}\diff{u}\\
					&=	\frac{z^{\alpha_1 -1} (1-z)^{\alpha_2 -1}}{\Gamma (\alpha _1) \Gamma (\alpha _2)} \Gamma(\alpha_1 + \alpha_2)\\
					&=	\frac{z^{\alpha_1 -1} (1-z)^{\alpha_2 -1}}{B(\alpha_1 +\alpha_2)} ;\alpha _1,\alpha _2>0, \text{ for } 0<z<1\\
		\end{split}
	\end{equation}
	This last result is clearly the beta distribution found in \autoref{eq:specialcases:beta:pdf}.
\end{proof}

It follows from the previous results that the beta distribution is, in fact, a proportion of two gamma distributions!

Here is a comparison of the histogram of the random variable $Z=\dfrac{X}{X+Y}$, where $X\sim\G\text{amma}(2, 1)$ and
$Y\sim\G\text{amma}(5, 1)$, with the plot of the probability density function of the beta distribution with parameters
$\B\text{eta}(2,5)$:
\figurecenter{1}{Simulation_quotient.png}{Comparison of the histogram of a quotient of gamma random variables with the probability density function of a beta distribution.}{fig:simulation:quotient}

%======================================================================================================================
%======================================================================================================================
%======================================================================================================================

\pagebreak
\subsection{The Chi-Squared Distribution}\label{subsec:specialcases:chi2}
Then, a gamma distribution with parameters $\alpha=\sfrac{n}{2}$ and $\lambda=\sfrac{1}{2}$ is called the
chi-squared distribution with $n$ degrees of freedom such that
\begin{equation}
	X\sim\G\text{ammma}(\sfrac{n}{2}, \sfrac{1}{2})\equiv\chi^2_n
\end{equation}
\begin{proof}
	First, to show that $\G\text{ammma}(\sfrac{n}{2},\sfrac{1}{2})\equiv\chi^2_n$, let's prove that
	$\G\text{ammma}(\sfrac{1}{2},\sfrac{1}{2})\equiv\chi^2_1$, a chi-squared distribution with one (1) degree of
	freedom, using a change of variables $Y=Z^2$ with $Z\sim\N(0, 1)$.
	\begin{equation}\label{eq:relation:chisquared:1}
		\begin{split}
			G(y)	&=	P(Y\leq y)\\
					&=	P(Z^2\leq y)\\
					&=	P(-\sqrt{y}\leq z \leq \sqrt{y})\\
					&=	\Phi(\sqrt{y})-\Phi(-\sqrt{y})\\[12pt]
			g(y)	&=	\frac{f(\sqrt{y})}{2\sqrt{y}}+\frac{f(-\sqrt{y})}{2\sqrt{y}}\\
					&=	\frac{e^{-\sfrac{y}{2}}}{2\sqrt{2\pi y}} + \frac{e^{-\sfrac{y}{2}}}{2\sqrt{2\pi y}}\\
					&=	\frac{\cancel{2}e^{-\sfrac{y}{2}}}{\cancel{2}\sqrt{2\pi y}}\\
					&=	\frac{1}{\sqrt{2}\Gamma(\frac{1}{2})}y^{-\sfrac{1}{2}}e^{-y\sfrac{1}{2}}\\
					&=	\frac{(\frac{1}{2})^{\sfrac{1}{2}}}{\Gamma(\frac{1}{2})}y^{\sfrac{1}{2}-1}e^{-(\sfrac{1}{2})y}
			\end{split}
	\end{equation}
	\begin{equation}
		\therefore Y\sim\G\text{ammma}(\sfrac{1}{2}, \sfrac{1}{2})\equiv\chi^2_1
	\end{equation}
	One may generalize this result by using $Z_1, Z_2, \ldots, Z_n$ as i.i.d. random variables following $\N(0, 1)$ to
	find the distribution of
	\begin{equation}
		Y = \sum^n_{i=1}Z^2_i
	\end{equation}
	Hence, using the moment generating function,
	\begin{equation}\label{eq:relation:chisquared:n}
		\begin{split}
			M_{\sum^n_{i=1}Z^2_i}(t)	&=	\prod^n_{i=1}M_{Z^2_i}(t)\\
										&=	M(t)^n\\
										&=	\left(\frac{\sfrac{1}{2}}{\sfrac{1}{2}-t}\right)^{\displaystyle\sfrac{n}{2}}
		\end{split}
	\end{equation}
	This last line is equal to the moment generating function of a random variable following the distribution
	$\G\text{amma}(\sfrac{n}{2}, \sfrac{1}{2})$. Hence,
	\begin{equation}
		\therefore Y\sim\G\text{amma}(\sfrac{n}{2}, \sfrac{1}{2})\equiv \chi^2_n
	\end{equation}
\end{proof}

Using the previously proven result, the density function of the chi-squared distribution may be determined by
substituting the parameters $\alpha = \sfrac{n}{2}$ and $\lambda = \sfrac{1}{2}$ into the density function of the gamma
distribution:
\begin{equation}\label{eq:chisquared-density}
	\begin{split}
		f(x)	&=	\frac{\lambda^\alpha}{\Gamma(\alpha)}x^{\alpha-1}e^{-\lambda x}\\
				&=	\frac{(\frac{1}{2})^{\sfrac{n}{2}}}{\Gamma(\frac{n}{2})}x^{\frac{n}{2}-1}e^{-\frac{1}{2}x}\\
				&=	\frac{1}{2^{\sfrac{n}{2}}\Gamma(\frac{n}{2})}x^{\frac{n}{2}-1}e^{-\sfrac{x}{2}}
	\end{split}
\end{equation}
\figurecenter{1}{Chi2_PDF.png}{Plot of the probability distribution function of a $\chi^2_n$ distribution for various values of $n$.}{fig:specialcases:chi2:pdf}

The chi-square distribution is often used to describe the distribution of a sum of squared random variables, to test the
fit of a distribution of data, or to determine whether data series are independent or dependent.

%======================================================================================================================
%======================================================================================================================
%======================================================================================================================

\pagebreak
\subsection{The Normal Distribution}\label{subsec:specialcases:normal}
As was shown previously in \autoref{subsec:specialcases:chi2}, the normal distribution is related to the gamma
distribution as it may be used to prove that the latter is related to the chi-squared distribution. That is:
\begin{equation}
	Z\sim\N\text{ormal}(0, 1)\Rightarrow Z^2\sim\G\text{amma}(\sfrac{1}{2}, \sfrac{1}{2})\equiv\chi^2_1
\end{equation}

Furthermore, since the gamma distribution $\G\text{amma}(\alpha, \lambda)$ may be viewed as a sum of $\alpha$ exponential
distributions, then by the central limit theorem, the density function of the gamma distribution will be extremely similar
to that of a normal distribution when $\alpha$ approaches $\infty$:
\begin{equation}
	\G\text{amma}(\alpha, \lambda) \approxequiv \N(\sfrac{\alpha}{\lambda}, \sfrac{\alpha}{\lambda^2})\text{ for }\alpha\to\infty
\end{equation}
The central limit theorem will not be proven as it is quite intuitive; the more data samples one has, the more
centred around the expected value the probability density function graph will be. That being said, here is a graph
representing this concept by comparing the probability density function graph of a $\G\text{amma}(60, 1)$ to that of a
$\N(60, 60)$. Because $\G\text{amma}(\alpha, \lambda)$ is in essence a sum of $\alpha$ i.i.d. random variables following
a distribution $\E\text{xp}(\lambda)$, then it makes sense that having a distribution $\G\text{amma}(60, 1)$ is like
having $60$ data samples following a $\E\text{xp}(1)$ distribution, which fits the central limit theorem.
\figurecenter{1}{Central_limit_theorem.png}{Graph showcasing the central limit theorem by comparing the probability density function plot of a $\G\text{amma}(60, 1)$ distribution to that of $\N(60, 60)$ distribution.}{fig:graph:clt}

%======================================================================================================================
%======================================================================================================================
%======================================================================================================================

\pagebreak
\subsection{The Erlang Distribution}\label{subsec:specialcases:erlang}
In real-world applications, it is often useful to limit the number of occurrences to a positive integer (it may not be
pertinent to determine, for example, the waiting time until 2.665 events take place). In this context, the special name
“erlang distribution” has been attributed to the specific gamma distribution in which the shape parameter $\alpha$ only
takes positive integer values and where the rate parameter $\lambda$ is a positive real number.
\begin{equation}
	f_{X_\alpha}(x)	= 	\begin{cases}
							\dfrac{\lambda^\alpha}{\Gamma(\alpha)}x^{\alpha-1}e^{-\lambda x} & \alpha\in\mathbb{N}; x\geq0\\
							0 & x<0
						\end{cases}
\end{equation}

This distribution takes its name from A. K. Erlang, the man who initially popularized its use in order to examine the
number of telephone calls which can be made at the same time to the operators of the switching stations, in the field of
phone traffic engineering. Indeed, the Erlang distribution's original purpose was to measure the time between incoming
calls, a statistic which can be used along with the expected duration of incoming calls to analyze phone traffic loads.
His work has since been expanded to consider queuing systems in general, as well as the load on web servers,
inter-purchase times and cancer incidence (see \autoref{sec:generalApplications})\cite{zachWhatErlangDistribution2020}.

%======================================================================================================================
%======================================================================================================================
%======================================================================================================================

\pagebreak
\subsection{The Wishart Distribution}
The Wishart distribution, named in honour of the Scottish mathematician John Wishart who first formulated it in 1928, is
a generalization to multiple dimensions of the gamma distribution: it is a family of distributions defined over symmetric
positive matrices. Namely, if $X_1, ..., X_n$ are independent and form a $p\times n$ matrix $X = (X_1, ..., X_n)$, then
the distribution of a $p\times n$ random matrix $M = XX^\prime = \sum^n_{i=1}X_iX^\prime_i$ is said to have a Wishart
distribution with $n$ degrees of freedom.

If $n\geq p$, such a matrix is denoted $\matr{M}\sim W_p(n, \Sigma)$ and has the following probability density function,
where $\Gamma_p(\alpha)$ is the multivariate gamma function and where $\Sigma$ is a fixed symmetric positive matrix of
size $p \times p$.
\begin{equation}
	f(\matr{M}) = \frac{1}{2^{\sfrac{np}{2}}\Gamma_p(\alpha)\left|\Sigma\right|^{\sfrac{n}{2}}}\left|\matr{M}\right|^{\sfrac{n-p-1}{2}}e^{-\frac{\text{trace}(\Sigma^{-1}\matr{M})}{2}}
\end{equation}

Interesting similarities with the probability density function of the gamma distribution can clearly be noticed in the
form of this equation. While this generalization of the gamma distribution to matrices is a rather intriguing topic that
was important to mention, this paper will not explore the Wishart distribution in greater depth
\cite{wikipediaWishartDistribution2022}.

%======================================================================================================================
%======================================================================================================================
%======================================================================================================================

\pagebreak
\section{Advanced Properties}
\subsection{Skewness}
In statistics, the skewness of a distribution is defined as the degree of asymmetry of the shape of the probability
distribution with respect to the normal distribution. Distributions can exhibit right (positive) or left (negative)
skewness; the normal distribution is qualified as exhibiting zero skewness. In the case of the gamma distribution, the
skewness is only dependent upon the shape parameter $\alpha$ and corresponds to\cite{chenLearnSkewness2021,
wikipediaGammaDistribution2022}:
\begin{equation}
	\text{Skewness} = \frac{2}{\sqrt{\alpha}}
\end{equation}

\subsection{Information Entropy}
The entropy of a random variable is defined as the average level of uncertainty associated with the possible outcomes
of its probability distribution. In the case of continuous probability distributions such as the gamma
distribution, one usually refers to differential entropy (also called continuous entropy), defined
as\cite{wikipediaDifferentialEntropy2022, wikipediaGammaDistribution2022}:
\begin{equation}
	h(X) = E[-\log(f(X))]
\end{equation}
More specifically, the entropy of the gamma distribution corresponds to:
\begin{equation}
	H(X) = \alpha + \ln(\lambda) + \ln(\Gamma(\alpha)) + (1-\alpha)\psi(\alpha)
\end{equation}
Following is the proof of this preceding equation:
\begin{proof}
	\begin{equation}
		\begin{split}
			H(X)	&=	E[-\ln(p(X))]\\
					&=	E[-\alpha\ln(\lambda) + \ln(\Gamma(\alpha)) - (\alpha-1)\ln(X) + \lambda X]\\
					&=	\alpha + \ln(\lambda) + \ln(\Gamma(\alpha)) + (1-\alpha)\psi(\alpha)
		\end{split}
	\end{equation}
\end{proof}

\subsection{Characteristic Function}\label{subec:advancedproperties:cf}
In statistics, any random variable associated with a probability density function is said to have a characteristic
function, which corresponds to the Fourier transformation of its probability density function. The characteristic
function thus simply corresponds to an alternative way of describing a random variable: it fully determines the behaviour
and properties of a given probability distribution to the same extent as the distribution function, but can provide
different insights to the features of the random variable. In the case of the gamma distribution, the characteristic
function is defined as\cite{wikipediaCharacteristicFunctionProbability2022}:
\begin{equation}
	\text{CF} = \left(1-\frac{it}{\lambda}\right)^{-\alpha}
\end{equation}

\subsection{Methods of Moments}
The "methods of moments" of a probability distribution are methods used to estimate population parameters, that is, to
describe various aspects of a given group or population. The methods of moments of the gamma distribution are defined
as\cite{wikipediaMethodMomentsStatistics2021, limWhatPopulationParameter2019}:
\begin{equation}
	\begin{split}
		\alpha	&=	\frac{E(X)^2}{\text{Var}(X)}\\
		\lambda	&=	\frac{E(X)}{\text{Var}(X)}
	\end{split}
\end{equation}

\subsubsection{Higher Moments}
To describe or estimate further shape parameters of a probability distribution, the higher-order moments of a function
can be used. Moments beyond 4th order are often increasingly hard to estimate, and larger sample sizes are often
required to obtain better quality of estimation\cite{wikipediaMomentMathematics2022}.

The $n$th raw moment is given by:
\begin{equation}
	E[X^n] = \frac{\Gamma(n+\alpha)}{\lambda^n\Gamma(\alpha)}
\end{equation}

\subsection{Excess kurtosis}
Kurtosis, the fourth standardized moment, is a characteristic used to describe the shape, or the “tailedness” of a
probability distribution. The normal distribution is used as the standard, displaying an excess kurtosis of zero with
which one can compare the excess kurtosis of other distributions\cite{wikipediaKurtosis2022}.

The exponential distribution, as a special case of the gamma distribution, has a positive excess kurtosis and is hence
called leptokurtic. This, in short, indicates that it exhibits fatter tails than the normal distribution. Following is
the equation characterizing the excess kurtosis of the gamma distribution\cite{wikipediaKurtosis2022}:
\begin{equation}
	\text{Kurtosis} = \frac{6}{\alpha}
\end{equation}

%======================================================================================================================
%======================================================================================================================
%======================================================================================================================

\pagebreak
\section{General Applications and Family}\label{sec:generalApplications}

The gamma distribution is used in many fields, since it is directly related to the Erlang, normal and exponential
distributions whose contributions extend to several disciplines. Here, we will present some of those applications.

\subsection{Insurance Companies}
First, the gamma distribution is of great use in the field of insurance services, given its direct relation to the
exponential distribution. For example, an analyst could use this distribution to specify the amount of time a product
lasts if one uses it at a constant average rate, thus modelling how reliable it is (and how much insurance should be
charged for it). Then, the size of loan defaults and the cost of insurance claims are also often modelled according to a
gamma distribution \cite{tiwariModelingInsuranceClaim2020}.

\subsection{Natural Events Prediction}
Then, the gamma distribution can also be used to model the amount of rainfall accumulated in a given reservoir. Indeed,
this distribution fits positive data, represents rainfall distribution well and its two parameters
give it sufficient flexibility to fit various climates\cite{husakUseGammaDistribution2007}.

\subsection{Customer Satisfaction}
\subsubsection{Service Time}
Service time can also be modelled using the gamma distribution. For example, if one is the sole person in line waiting for a meal, the
waiting time until one receives the long-awaited food can be modelled using an exponential distribution. Using the same
principle, the Erlang distribution can allow one to determine the total length of a process, that is, the duration of a
sequence of independent events. For instance, if a large number of people are waiting in line to be served, the
distribution of each of their individual waiting times (the sum of several independent exponentially distributed
variables) will correspond to the time it takes for the employee to serve everyone in it. Therefore, the gamma
distribution lies at the heart of what is called queuing theory: the mathematical study of the congestion of waiting
lines. Since waiting lines are found in countless places such as banks, restaurants and hospitals, as well as on web
servers or multistep manufacturing and distribution processes, the gamma distribution provides very useful applications
in everyday life.

\subsubsection{Call Centers}
One of the most famous of those applications concerns phone queuing, on which A. K. Erlang famously worked. The Erlang
distribution has indeed been developed in the objective to model the time between incoming calls at a call centre, along
with the expected number of calls, thus allowing call centres to know what their staffing capacity should be depending
on the time of day.

\subsubsection{Retail}
Beyond waiting lines, the Erlang distribution is often used by retailers to model the frequency of inter-purchase times
by consumers, which gives them an idea of how often a given consumer is expected to purchase a product from them and
helps them control inventory and staffing.

\subsection{Oncology}
Finally, in the field of oncology, the age distribution of cancer incidence also follows a gamma distribution. Although
the factors underlying cancer development are not yet fully understood, it has been hypothesized that cancers arise
after several successive “driver events”, that is, after some number of mutations occurs in a cell. Analyses of cancer
statistics suggest that the incidence of the most prevalent cancer types with respect to the patients' age closely
follows the gamma probability distribution or, more specifically, the Erlang distribution. This may be due to the fact
that, more broadly, the Erlang distribution can be used to model the cell cycle time
distribution\cite{belikovNumberKeyCarcinogenic2017}. The age distribution of cancer can be modelled using the following equation:

\begin{equation}
	Y = A\cdot x^{\alpha-1}\frac{e^{\frac{-x}{\beta}}}{\beta^\alpha}\cdot\Gamma(k)
\end{equation}

In this case, the shape parameter $\alpha$ predicts the number of carcinogenic driver events, whereas the scale
parameter $\beta$ predicts the average time between those events for each cancer type. Using an additional amplitude
parameter A, the maximal population susceptibility to a given type of cancer can even be
predicted\cite{belikovNumberKeyCarcinogenic2017}. Given that experimental research on cancer development is crucial for
the lives of many people, numerical references such as that provided by the gamma distribution are of paramount
importance in our society. The gamma distribution can save lives, if it is used wisely!

\figurecenter{1}{Cancer_probability}{"The Erlang distribution approximates cancer incidence by age for 20 most prevalentcancer types. Dots indicate actual data for 5-year age intervals, curves indicate the PDF of the Erlang distribution fitted to the data [...]. The middle age of each age group is plotted. Cancer types are arranged in the order of decreasing incidence"\cite{belikovNumberKeyCarcinogenic2017}.}{fig:cancerProbability}

%======================================================================================================================
% END OF DOCUMENT =====================================================================================================
%======================================================================================================================
\pagebreak

\begin{appendix}
\section{Graphs - Python Code}
\subsection{Probability Density Function}
Here is the code used to generate the probability density function graph of the gamma distribution in
\autoref{subsec:gamma:pdf}:
\lstinputlisting[language=Python, caption={}, label=code:gamma:pdf]{_assets/Gamma_pdf.py}

\pagebreak
\subsection{Cumulative Distribution Function}
Here is the code used to generate the cumulative distribution function graph of the gamma distribution in
\autoref{subsec:gamma:cdf}:
\lstinputlisting[language=Python, caption={}, label=code:gamma:cdf]{_assets/Gamma_cdf.py}

\pagebreak
\subsection{Central Limit Theorem}
Here is the code used to generate the central limit theorem graph in \autoref{subsec:specialcases:normal}:
\lstinputlisting[language=Python, caption={}, label=code:gamma:clt]{_assets/Central_limit_theorem.py}

\pagebreak
\section{Simulations - Python Code}
\subsection{Simulation 1: $X+Y$}
Comparison of the graph of $\G\text{amma}(\alpha+\beta, \lambda)$ and the histogram of $X+Y$ with
$X\sim\G\text{amma}(\alpha, \lambda)$ and $Y\sim\G\text{amma}(\beta, \lambda)$. Featured in
\autoref{fig:fig:simulation:sum}.
\lstinputlisting[language=Python, caption={}, label=code:simulation:sum]{_assets/Simulation_sum.py}

\pagebreak
\subsection{Simulation 2: $\frac{X}{X+Y}$}
Comparison of the graph of $\B\text{eta}(\alpha, \beta)$ and the histogram of $\frac{X}{X+Y}$ with
$X\sim\G\text{amma}(\alpha, \lambda)$ and $Y\sim\G\text{amma}(\beta, \lambda)$. Featured in
\autoref{fig:simulation:quotient}.
\lstinputlisting[language=Python, caption={}, label=code:simulation:quotient]{_assets/Simulation_quotient.py}

\pagebreak
\phantomsection\listoffigures
\end{appendix}

\pagebreak\phantomsection\printbibliography[heading=bibintoc, title={References}]
\end{document}

\begin{tabular}
