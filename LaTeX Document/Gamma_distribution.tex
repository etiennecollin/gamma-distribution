\input{/Users/etiennecollin/Programming/LaTex/Paths}
\path{macOS}

\documentclass[12pt]{article}
\usepackage[english]{babel}																												% Selects language of document
\usepackage{./_assets/Packages}																											% Imports custom packages selection
\usepackage[style=numeric,autocite=superscript,sorting=none,backref=false,backend=biber,date=iso,datezeros=true,seconds=true]{biblatex}	% Bibliography options. autocite=<inline, footnote, superscript, plain>
% \usepackage{indentfirst}																												% Used to indent first paragraph of section

%======================================================================================================================
% DOCUMENT USER SETTINGS ==============================================================================================
%======================================================================================================================
\newcommand{\docAuthorName}{Etienne Collin, Marie Ouellet \& Rania Yahyaoui}
\newcommand{\docAuthorStudentNumber}{}
\newcommand{\docAuthorTitlePage}{\docAuthorName}
\newcommand{\docClass}{Probability and Statistics}
\newcommand{\docClassInstructor}{Professor Vincent Carrier}
\newcommand{\docClassNumber}{201-BNM-LW}
\newcommand{\docClassTime}{Section 20108}
\newcommand{\docClassSemester}{Winter 2022}
\newcommand{\docDueDate}{May 25, 2022}
\newcommand{\docDueTime}{23:59}
\newcommand{\docSubTitle}{Properties, Proofs and Applications}
\newcommand{\docTitle}{The Gamma Distribution}
\usepackage[svgnames, hyperref]{xcolor}	% See p.42 https://mirror.its.dal.ca/ctan/macros/latex/contrib/xcolor/xcolor.pdf

% Page settings
\topmargin=-0.45in
\evensidemargin=0in
\oddsidemargin=0in
\textwidth=6.5in
\textheight=9.0in
\headsep=0.25in
\linespread{1.1}
\pagestyle{fancy}
\fancyhf[HL]{\firstxmark}						% Content on left of head
\fancyhf[HC]{\docTitle}						% Content on center of head
\fancyhf[HR]{Collin – \docAuthorStudentNumber}			% Content on right of head
\fancyhf[FL]{\docClassInstructor}					% Content on left of foot
\fancyhf[FC]{\thepage}						% Content on center of foot
\fancyhf[FR]{\lastxmark}						% Content on right of foot
\renewcommand\headrulewidth{0.4pt}				% Width of line separation head from body
\renewcommand\footrulewidth{0.4pt}				% Width of line separation foot from body
% \setlength\parindent{5mm}					% Custom Paragraph indentation
\setlength\parindent{0.5in}					% Custom Paragraph indentation											% Imports custom page settings
% DEFINITIONS =========================================================================================================
% Paths
\addbibresource{\basepath/bibliography.bib}	% Imports bibliography file
\graphicspath{{images/}}

% lstlistings setup ---------------------------------------------------------------------------------------------------
\definecolor{commentsColor}{gray}{0.6}
\definecolor{keywordsColor}{RGB}{249,38,114}
\definecolor{stringColor}{RGB}{230, 219, 116}
\definecolor{mygreen}{rgb}{0,0.6,0}
\definecolor{mygray}{rgb}{0.5,0.5,0.5}
\definecolor{mymauve}{rgb}{0.58,0,0.82}
\lstset{
  backgroundcolor=\color{white},   % choose the background color; you must add \usepackage{color} or \usepackage{xcolor}; should come as last argument
  basicstyle=\footnotesize,        % the size of the fonts that are used for the code
  breakatwhitespace=false,         % sets if automatic breaks should only happen at whitespace
  breaklines=true,                 % sets automatic line breaking
  captionpos=b,                    % sets the caption-position to bottom
  commentstyle=\color{mygreen},    % comment style
  deletekeywords={},            % if you want to delete keywords from the given language
  escapeinside={\%*}{*)},          % if you want to add LaTeX within your code
  extendedchars=true,              % lets you use non-ASCII characters; for 8-bits encodings only, does not work with UTF-8
  firstnumber=0,                % start line enumeration with line 0
  frame=single,	                   % adds a frame around the code
  keepspaces=true,                 % keeps spaces in text, useful for keeping indentation of code (possibly needs columns=flexible)
  keywordstyle=\color{blue},       % keyword style
  language=Python,                 % the language of the code
  morekeywords={*},            % if you want to add more keywords to the set
  numbers=left,                    % where to put the line-numbers; possible values are (none, left, right)
  numbersep=5pt,                   % how far the line-numbers are from the code
  numberstyle=\tiny\color{mygray}, % the style that is used for the line-numbers
  rulecolor=\color{black},         % if not set, the frame-color may be changed on line-breaks within not-black text (e.g. comments (green here))
  showspaces=false,                % show spaces everywhere adding particular underscores; it overrides 'showstringspaces'
  showstringspaces=false,          % underline spaces within strings only
  showtabs=false,                  % show tabs within strings adding particular underscores
  stepnumber=2,                    % the step between two line-numbers. If it's 1, each line will be numbered
  stringstyle=\color{mymauve},     % string literal style
  tabsize=4,	                   % sets default tabsize to 2 spaces
  title=\lstname                   % show the filename of files included with \lstinputlisting; also try caption instead of title
}

% Sections alignment --------------------------------------------------------------------------------------------------
% Gets label of current section
\makeatletter\newcommand{\currentlabel}{\@currentlabelname}\makeatother
% Sets alignment of sections
\newcommand{\formatsection}[1]{
    \ifthenelse{\equal{#1}{center}}{
		%Makes sections centered
		\titleformat{\section}{\normalfont\Large\bfseries\centering}{\thesection}{1em}{}
    }{
        \ifthenelse{\equal{#1}{right}}{
			%Makes sections right-aligned
			\titleformat{\section}{\normalfont\Large\bfseries\raggedleft}{\thesection}{1em}{}
        }{
			\ifthenelse{\equal{#1}{left}}{
				%Makes sections left-aligned
				\titleformat{\section}{\normalfont\Large\bfseries\raggedright}{\thesection}{1em}{}
        	}{
        	}
        }
    }
}

% tabularx
\newcolumntype{Y}{>{\centering\arraybackslash}X}

% Caption/Ref settings -------------------------------------------------------------------------------------------------
\captionsetup{
    format = hang,
    labelfont = sc,
	skip = 6pt,
	hypcapspace = 96pt
}

% TOC & Sections -------------------------------------------------------------------------------------------------------
% Sets TOC depth and change changes its display name
\setcounter{secnumdepth}{3} % Change to 0 for unnumbered sections
\addto\captionsenglish{
  \renewcommand{\contentsname}{Table of Contents}
}
% Makes the paragraph command appear like a subsubsubsection
\titleformat{\paragraph}{\normalfont\normalsize\bfseries}{\theparagraph}{1em}{}
\titlespacing*{\paragraph}{0pt}{3.25ex plus 1ex minus .2ex}{1.5ex plus .2ex}

% SIunitx -------------------------------------------------------------------------------------------------------------
% Sets \per to display as frac
\sisetup{per-mode=symbol-or-fraction}
\DeclareSIUnit\Molar{\mole\per\liter}
\DeclareSIUnit\molar{\textsc{M}}
\DeclareSIUnit\clight{\text{\ensuremath{c}}_{0}}

% BibLaTex ------------------------------------------------------------------------------------------------------------
% BibLaTex definitions
\let\oldcite\cite
\renewcommand{\cite}{\autocite}
\let\oldsupercite\supercite
\renewcommand{\supercite}[1]{\textsuperscript{[}\oldsupercite{#1}\textsuperscript{]}}
% Makes number created by \supercite to be between brackets.
\DeclareCiteCommand{\supercite}[\mkbibsuperscript]{%
	\iffieldundef{prenote}{}{\BibliographyWarning{Ignoring prenote argument}}%
	\iffieldundef{postnote}{}{\BibliographyWarning{Ignoring postnote argument}}%
}
{\usebibmacro{citeindex}\,\bibopenbracket\usebibmacro{cite}\bibclosebracket}
{\supercitedelim}{}

% BASIC DOCUMENT SETTINGS AND TITLE PAGE ==============================================================================
% PDF export settings
\hypersetup{
	pdftex,
	pdfauthor = {Etienne Collin},
	bookmarksnumbered = true,
	bookmarksopen = true,
	bookmarksopenlevel = 1,
	pdfstartview = Fit,
	pdfpagemode = UseOutlines,
	pdfpagelayout = TwoPageRight,
	colorlinks = true,		% Colors links instead of ugly boxes
	urlcolor = DarkBlue,	% Color for external hyperlinks
	linkcolor = black,		% Color of internal links
	citecolor = black,		% Color of citations
	filecolor = black,		% Color of file links
}

% Custom font

\newcommand{\customfont}[1]{
	\usepackage{mathspec}	% Enable the power of CUSTOM FONTS with XeLaTeX
	\setallmainfonts(Digits,Latin,Greek)[Numbers={Lining,Proportional}]{#1}
	\setallmonofonts[Numbers={Lining,Proportional}]{#1}
	\setallsansfonts[Numbers={Lining,Proportional}]{#1}
	\setmathsfont(Digits,Latin,Greek)[Numbers={Lining,Proportional}]{#1}
	\setmathsf[Numbers={Lining,Proportional}]{#1}
	\setmathbb[Numbers={Lining,Proportional}]{#1}
	\setmathcal[Numbers={Lining,Proportional}]{#1}
}

% ENVIRONMENTS ========================================================================================================
% hmwkProblem ---------------------------------------------------------------------------------------------------------
\newcounter{partCounter}
\renewcommand{\part}[1]{\noindent\textbf{\large Part (\alph{partCounter})}\stepcounter{partCounter}\\}
\newcounter{hmwkProblemCounter}
\setcounter{hmwkProblemCounter}{1}

% Continued Header Changer
\newcommand{\enterProblemHeader}[1]{
	\nobreak\extramarks{}{}\nobreak{}
	\nobreak\extramarks{Problem \arabic{#1} (continued)}{Problem \arabic{#1} continued on next page\ldots}\nobreak{}
}
\newcommand{\exitProblemHeader}[1]{
	\nobreak\extramarks{Problem \arabic{#1} (continued)}{Problem \arabic{#1} continued on next page\ldots}\nobreak{}
	\stepcounter{#1}
	\nobreak\extramarks{}{}\nobreak{}
}

% Environment. Optional argument will adjust the problem counter.
\newenvironment{hmwkProblem}[1][-1]{
	\ifnum#1>0
		\setcounter{hmwkProblemCounter}{#1}
	\fi
	\pagebreak
	\section{Problem \arabic{hmwkProblemCounter}}
	\stepcounter{section}
	\setcounter{partCounter}{1}
	\enterProblemHeader{hmwkProblemCounter}
}{
	\exitProblemHeader{hmwkProblemCounter}
}

% sect ----------------------------------------------------------------------------------------------------------------
% Will write "continued" when content spans >1 page. Argument adjusts the name of the section.
\newenvironment{sect}[1]{
	\pagebreak
	\section{#1}
	\stepcounter{section}
	\setcounter{partCounter}{1}
	\nobreak\extramarks{}{}\nobreak{}
	\nobreak\extramarks{\currentlabel\ (continued)}{\currentlabel\ continued on next page\ldots}\nobreak{}
}{
	\nobreak\extramarks{\currentlabel\ (continued)}{\currentlabel\ continued on next page\ldots}\nobreak{}
	\nobreak\extramarks{}{}\nobreak{}
}

% SETUP/ALIASES =======================================================================================================
\newcommand{\imagecenter}[2][1]{
	\begin{center}
		\includegraphics[width=\textwidth*{#1}]{#2}
	\end{center}
}
\newcommand{\imageleft}[2][1]{
	\begin{flushleft}
		\includegraphics[width=\textwidth*{#1}]{#2}
	\end{flushleft}
}
\newcommand{\imageright}[2][1]{
	\begin{flushright}
		\includegraphics[width=\textwidth*{#1}]{#2}
	\end{flushright}
}
\newcommand{\figurecenter}[4]{
	\begin{figure}[H]
		\centering
		\begin{minipage}[H]{#1\textwidth}
			\centering
			\includegraphics[width=\textwidth]{#2}
			\caption{#3}
			\label{#4}
		\end{minipage}
	\end{figure}
}
\newcommand{\figuredual}[6]{
	\begin{figure}[H]
		\begin{minipage}[H]{0.48\textwidth}
			\centering
			\includegraphics[width=\textwidth]{#1}
			\raggedleft
			\caption{#2}
			\label{#3}
		\end{minipage}
		\hfill
		\begin{minipage}[H]{0.48\textwidth}
			\centering
			\includegraphics[width=\textwidth]{#4}
			\caption{#5}
			\label{#6}
		\end{minipage}
	\end{figure}
}
\newcommand{\figuredualsingle}[6]{
	\begin{figure}[H]
		\begin{minipage}[H]{0.48\textwidth}
			\centering
			\includegraphics[width=\textwidth]{#1}
			\subcaption{#2}\label{#6-a}
		\end{minipage}
		\hfill
		\begin{minipage}[H]{0.48\textwidth}
			\centering
			\includegraphics[width=\textwidth]{#3}
			\subcaption{#4}\label{#6-b}
		\end{minipage}
		\caption{#5}\label{#6}
	\end{figure}
}

% Theorems
\theoremstyle{plain}
\newtheorem{definition}{Definition}[subsection]
\newtheorem{theorem}{Theorem}[subsection]
\newtheorem{example}{Example}[theorem]
\newtheorem{corollary}{Corollary}[subsection]
\newtheorem{lemma}{Lemma}[theorem]
\newtheorem{proposition}{Proposition}[theorem]
% Unnumbered version
\newtheorem*{definition*}{Definition}
\newtheorem*{theorem*}{Theorem}
\newtheorem*{example*}{Example}
\newtheorem*{corollary*}{Corollary}
\newtheorem*{lemma*}{Lemma}
\newtheorem*{proposition*}{Proposition}

\theoremstyle{definition}
\newtheorem{exercise}{Exercise}[subsection]
% Unnumbered version
\newtheorem*{exercise*}{Exercise}


\theoremstyle{remark}
\newtheorem{remark}{Remark}
\newtheorem{case}{Case}
\newtheorem{claim}{Claim}[subsection]
% Unnumbered version
\newtheorem*{remark*}{Remark}
\newtheorem*{case*}{Case}
\newtheorem*{claim*}{Claim}

\renewcommand\qedsymbol{QED}

% Useful shortcuts
\newcommand{\alg}[1]{\textsc{\bfseries\footnotesize #1}}
\newcommand{\deriv}[2][]{\mathop{}\!\frac{\mathrm{d}^{#1}}{\mathrm{d}#2^{#1}}}	% For derivatives
\newcommand{\pderiv}[3][]{\mathop{}\!\dfrac{\partial^{#1}#2}{\partial#3^{#1}}}					% For partial derivatives
\newcommand{\pdiff}[2][]{\mathop{}\!\dfrac{\partial^{#1}}{\partial#2^{#1}}}					% For partial derivatives
\newcommand{\diff}[2][]{\mathop{}\!\mathrm{d^{#1}}#2}							% Integral dx
\newcommand*{\widefbox}[1]{\fbox{\hspace{2em}#1\hspace{2em}}}						% For boxed align environment

% Enables new list environment that supports nesting
\newlist{legal}{enumerate}{10}
\setlist[legal]{label*=.\arabic*}
\setlist[legal,1]{label=\arabic*}

% Temporary text markers
\newcommand{\solution}{\noindent\textbf{\large Solution}}				% Alias for the Solution section header
\newcommand{\missing}{\textcolor{red}{\textbf{VALUE MISSING}}}	% Creates a big red "Missing Value" string.
\newcommand{\inprogress}{\textcolor{red}{\textbf{IN PROGRESS}}}	% Creates a big red "In Progress" string.
\newcommand{\incomplete}{\textcolor{red}{\textbf{INCOMPLETE}}}	% Creates a big red "Incompletes" string.
\newcommand{\bigred}[1]{\textcolor{red}{\textbf{#1}}}	% Creates a big red string.

% END OF SETUP DOCUMENT ===============================================================================================											% Imports custom environments and definitions
\fancyhf[HR]{\docClassTime}													% Removes student number from right header
\newcommand{\G}{\mathcal{G}}
\newcommand{\E}{\mathcal{E}}
\newcommand{\N}{\mathcal{N}}
%======================================================================================================================
% SOURCE ==============================================================================================================
%======================================================================================================================
\begin{document}
\singlespacing
% Title page
\begin{titlepage}
	\begin{center}
		\vspace*{0.5in}
		\LARGE{\textsc{\docTitle}}\\
		\vspace*{0.1in}
		\Large{\textsc{\docSubTitle}}\\
		\vfill
		%
		\Large{\textsc{\docAuthorTitlePage}}\\
		\vspace*{0.1in}
		\Large{\textsc{\docClass\ –\ \docClassNumber}}\\
		\vfill
		%
		\Large{\docClassTime}\\
		\vspace*{0.1in}
		\large{\textsc{\docClassInstructor}}\\
		\vfill
		%
		\large{\textsc{St. Lawrence Cégep Champlain}}\\
		\vspace*{0.1in}
		\small{Due on \docDueDate\ at \docDueTime}\\
		\vfill
	\end{center}
\end{titlepage}
\pagebreak
\phantomsection\tableofcontents\pagebreak
\doublespacing
%======================================================================================================================
% START OF DOCUMENT ===================================================================================================
%======================================================================================================================
\section{General Applications and Family}\label{section:cancer}

The gamma distribution is used in many fields, since it is directly related to the Erlang, normal and exponential
distributions whose contributions extend to several disciplines. Here, we will present some of those applications.

\subsection{Insurance Companies}
First, the gamma distribution is of great use in the field of insurance services, given its direct relation to the
exponential distribution. For example, an analyst could use this distribution to specify the amount of time a product
lasts if one uses it at a constant average rate, thus modeling how reliable it is (and how much insurance should be
charged for it). Then, the size of loan defaults and the cost of insurance claims are also often modeled according to a
gamma distribution.

\subsection{Natural Events Prediction}
Then, the gamma distribution can also be used to model the amount of rainfall accumulated in a given reservoir. Indeed,
this distribution fits positive data, represents rainfall distribution well and its two parameters -shape and scale-
give it sufficient flexibility to fit various climates\cite{husakUseGammaDistribution2007a}.

\subsection{Customer Satisfaction}
Service time can also be modeled using the gamma distribution. For example, if one is waiting in line for a meal, the
waiting time until one receives the long-awaited food can be modelled using an exponential distribution. Using the same
principle, the Erlang distribution can allow one to determine the total length of a process, that is, the duration of a
sequence of independent events. For instance, if a large number of people are waiting in line to be served, the
distribution of each of their individual waiting times (the sum of several independent exponentially distributed
variables) will correspond to the time it takes for the employee to serve everyone in it. Therefore, the gamma
distribution lies at the heart of what is called queuing theory: the mathematical study of the congestion of waiting
lines. Since waiting lines are found in countless places such as banks, restaurants and hospitals, as well as on web
servers or multistep manufacturing and distribution processes, the gamma distribution provides very useful applications
in everyday life.

\subsection{Call Centers}
One of the most famous of those applications concerns phone queuing, on which A. K. Erlang famously worked. The Erlang
distribution has indeed been developed in the goal to model the time in between incoming calls at a call center, along
with the expected number of calls, thus allowing call centers to know what their staffing capacity should be depending
on the time of day.

Beyond waiting lines, the Erlang distribution is often used by retailers to model the frequency of interpurchase times
by consumers, which gives them an idea of how often a given consumer is expected to purchase a product from them and
helps them control inventory and staffing.

\subsection{Oncology}
Finally, in the field of oncology, the age distribution of cancer incidence also follows a gamma distribution. Although
the factors underlying cancer development are not yet fully understood, it has been hypothesized that cancers arise
after several successive “driver events”, that is, after some number of mutations occurs in a cell. Analyses of cancer
statistics suggest that the incidence of the most prevalent cancer types with respect to the patients' age closely
follows the gamma probability distribution or, more specifically, the Erlang distribution. This may be due to the fact
that, more broadly, the Erlang distribution can be used to model cell cycle time
distribution\cite{belikovNumberKeyCarcinogenic2017} .

\begin{equation*}
	Y = A\cdot x^{k-1}\frac{e^{\frac{-x}{b}}}{b^k}\cdot\Gamma(k)
\end{equation*}

In this case, the shape parameter $\alpha$ predicts the number of carcinogenic driver events, whereas the shape
parameter $\lambda$ predicts the average time between those events for each cancer type. Using an additional amplitude
parameter A, the maximal population susceptibility to a given type of cancer can even be
predicted\cite{belikovNumberKeyCarcinogenic2017}. Given that experimental research on cancer development is crucial for
the lives of many people, numerical references such as that provided by the gamma distribution are of paramount
importance in our society. The gamma distribution can save lives, if it is used wisely!

\figurecenter{1}{Cancer_probability}{"The Erlang distribution approximates cancer incidence by age for 20 most prevalent cancer types. Dots indicate actual data for 5-year age intervals, curves indicate the PDF of the Erlang distribution fitted to the data [...]. The middle age of each age group is plotted. Cancer types are arranged in the order of decreasing incidence"\cite{belikovNumberKeyCarcinogenic2017} .}{fig:cancerProbability}

%======================================================================================================================
%======================================================================================================================
%======================================================================================================================

\pagebreak
\section{Special Cases of the Gamma Distribution}
The gamma distribution has many common parameterizations. Hence, in order to make working with them easier, they were
given a specific name such as the chi-square distribution, the exponential distribution and more. In this section, we
will present some of these special cases and prove their relationship.

\subsection{The Erlang Distribution}
In real world applications, it is often useful to limit the number of occurrences to a positive integer (it may not be
pertinent to determine, for example, the waiting time until 2.665 events take place). In this context, the special name
“Erlang distribution” has been attributed to the specific gamma distribution in which the shape parameter $\alpha$ only
takes positive integer values and where the rate parameter $\lambda$ is a positive real number.
\begin{equation*}
	f_{X_\alpha}(x)	= 	\begin{cases}
							\dfrac{\lambda^\alpha}{\Gamma(\alpha)}x^{\alpha-1}e^{-\lambda x}  & \alpha\in\mathbb{Z}^+; x\geq0\\
							0  & x<0
  						\end{cases}
\end{equation*}

This distribution takes its name from A. K. Erlang, the man who initially popularized its use in order to examine the
number of telephone calls which can be made at the same time to the operators of the switching stations, in the field of
phone traffic engineering. Indeed, the Erlang distribution's original purpose was to measure the time between incoming
calls, a statistic which can be used along with the expected duration of incoming calls to analyze phone traffic loads.
His work has since been expanded to consider queuing systems in general, as well as the load on web servers,
interpurchase times and cancer incidence (see \autoref{section:cancer})\cite{zachWhatErlangDistribution2020}.

%======================================================================================================================
%======================================================================================================================
%======================================================================================================================

\pagebreak
\subsection{The Exponential Distribution}
The gamma distribution can model the elapsed time between random and independent events. However, what should one do
when they want to strictly model the time to wait before a new event?

If a unique event is modeled, it follows that the shape parameter is limited to $\alpha = 1$. In such a case, the gamma
distribution is said to follow an exponential distribution: more specifically, if $X\sim\G\text{amma}(1, \lambda)$, then
it can also be said that $X\sim\E\text{xp}(\lambda)$. $X$ thus follows an exponential distribution with rate parameter
$\lambda$, where the rate parameter represents how quickly events occur. More concretely:
\begin{equation*}
	X\sim\G\text{amma}(1, \lambda)\equiv\E\text{xp}(\lambda)
\end{equation*}
\bigred{MAYBE HERE INCLUDE GRAPH OF GAMMA DISTRIBUTION WITH SHAPE PARAMETER 1  COMPARED TO OTHER SHAPE PARAMETERS ?}
Indeed, using a simple substitution of the variables, one gets that:
\begin{equation*}\label{eq:relation:exp}
	\begin{split}
		f(x)	&=	\frac{\lambda^\alpha x^{\alpha-1}}{\Gamma(\alpha)}e^{-\lambda x}\\
				&=	\frac{\lambda^1 x^{0}}{\Gamma(1)}e^{-\lambda x}\\
				&=	\lambda e^{-\lambda x}
	\end{split}
\end{equation*}
\begin{equation*}
	\therefore X\sim\G\text{amma}(1, \lambda)\equiv X\sim\E\text{xp}(\lambda)
\end{equation*}
Accordingly, it can be demonstrated that the sum of exponential random variables is a gamma random variable. That is:
\begin{equation*}
	\text{If } X_1,\,X_2,\,\ldots,\,X_n \text{ are i.i.d random variables following } \E\text{xp}(\lambda)\text{, then }Y=\sum^k_{i=1}\left[X_i\right]\sim\G\text{amma}(k, \lambda)
\end{equation*}
\begin{proof}
	\bigred{ADD HERE PROOF FOR THE DENSITY FUNCTION OF THE SUM OF EXPONENTIAL RANDOM VARIABLES (ETIENNE'S PART)}
	\begin{equation*}
		XYZ
	\end{equation*}
\end{proof}

%======================================================================================================================
%======================================================================================================================
%======================================================================================================================

\pagebreak
\subsection{The Chi-Squared Distribution}
Then, the gamma distribution with parameters $\alpha=\sfrac{n}{2}$ and $\lambda=\sfrac{1}{2}$ is called the
chi-squared distribution with $n$ degrees of freedom such that
\begin{equation*}
	X\sim\G\text{ammma}(\sfrac{n}{2}, \sfrac{1}{2})\equiv\chi^2_n
\end{equation*}
\begin{proof}
	First, to show that $\G\text{ammma}(\sfrac{n}{2},\sfrac{1}{2})\equiv\chi^2_n$, let's prove that
	$\G\text{ammma}(\sfrac{1}{2},\sfrac{1}{2})\equiv\chi^2_1$, a chi-squared distribution with one (1) degree of
	freedom, using a change of variables with $Y=Z^2$ for $Y\sim\chi^2_1$.
	\begin{equation*}\label{eq:relation:chisquared:1}
		\begin{split}
			G(y)	&=	P(Y\leq y)\\
					&=	P(Z^2\leq y)\\
					&=	P(-\sqrt{y}\leq z \leq \sqrt{y})\\
					&=	\Phi(\sqrt{y})-\Phi(-\sqrt{y})\\[12pt]
			g(y)	&=	\frac{f(\sqrt{y})}{2\sqrt{y}}+\frac{f(-\sqrt{y})}{2\sqrt{y}}\\
					&=	\frac{e^{-\sfrac{y}{2}}}{2\sqrt{2\pi y}} + \frac{e^{-\sfrac{y}{2}}}{2\sqrt{2\pi y}}\\
					&=	\frac{\cancel{2}e^{-\sfrac{y}{2}}}{\cancel{2}\sqrt{2\pi y}}\\
					&=	\frac{1}{\sqrt{2}\Gamma(\frac{1}{2})}y^{-\sfrac{1}{2}}e^{-y\sfrac{1}{2}}\\
					&=	\frac{(\frac{1}{2})^{\sfrac{1}{2}}}{\Gamma(\frac{1}{2})}y^{\sfrac{1}{2}-1}e^{-(\sfrac{1}{2})y}
			\end{split}
	\end{equation*}
	\begin{equation*}
		\therefore Y\sim\G\text{ammma}(\sfrac{1}{2}, \sfrac{1}{2})\equiv\chi^2_1
	\end{equation*}
	One may generalize this result by using that $Z_1,\,Z_2,\,\ldots,\,Z_n$ are i.i.d. random variables following $\N(0,
	1)$ to find the distribution of
	\begin{equation*}
		Y = \sum^n_{i=1}Z^2_i
	\end{equation*}
	Hence, using the moment generating function,
	\begin{equation*}\label{eq:relation:chisquared:n}
		\begin{split}
			M_{\sum^n_{i=1}Z^2_i}(t)	&=	\prod^n_{i=1}M_{Z^2_i}(t)\\
										&=	M(t)^n\\
										&=	\left(\frac{\sfrac{1}{2}}{\sfrac{1}{2}-t}\right)^{\displaystyle\sfrac{n}{2}}
		\end{split}
	\end{equation*}
	This last line is equal to the moment generating function of a random variable following the distribution
	$\G\text{amma}(\sfrac{n}{2}, \sfrac{1}{2})$. Hence,
	\begin{equation*}
		\therefore Y\sim\G\text{amma}(\sfrac{n}{2}, \sfrac{1}{2})\equiv \chi^2_n
	\end{equation*}
\end{proof}

Using the result proved previously, the density function of the chi-squared distribution may be determined by
substituting the parameters $\alpha = \sfrac{n}{2}$ and $\lambda = \sfrac{1}{2}$ into the density function of the gamma
distribution:
\begin{equation*}\label{eq:chisquared-density}
	\begin{split}
		f(x)	&=	\frac{\lambda^\alpha}{\Gamma(\alpha)}x^{\alpha-1}e^{-\lambda x}\\
				&=	\frac{(\frac{1}{2})^{\sfrac{n}{2}}}{\Gamma(\frac{n}{2})}x^{\frac{n}{2}-1}e^{-\frac{1}{2}x}\\
				&=	\frac{1}{2^{\sfrac{n}{2}}\Gamma(\frac{n}{2})}x^{\frac{n}{2}-1}e^{-\sfrac{x}{2}}
	\end{split}
\end{equation*}

\bigred{HERE INCLUDE GRAPH OF CHI SQUARE DISTRIBUTIONS}

The chi square distribution is often used to describe the distribution of a sum of squared random variables, to test the
fit of a distribution of data or to determine whether data series are independent or dependent.

\subsection{The Normal Distribution}
Text on the normal distribution

%======================================================================================================================
%======================================================================================================================
%======================================================================================================================

\pagebreak
\subsection{The Beta Distribution}
Text on the Beta distribution

%======================================================================================================================
%======================================================================================================================
%======================================================================================================================

\pagebreak
\subsection{The Wishart Distribution}
Text on the Wishart distribution

%======================================================================================================================
%======================================================================================================================
%======================================================================================================================

\pagebreak
\section{Introduction}
The gamma distribution is part of the two-parameters family of continuous probability distributions. Indeed, it may be
parameterized with two different parameterizations\cite{wikipediaGammaDistribution2022} :

\noindent Parameterization 1:
\vspace*{-24pt}
\begin{align}\label{eq:gamma:parameterization:1}
	\text{Shape: } \alpha>0		&&	\text{Rate: }\lambda>0
\end{align}

\noindent Parameterization 2:
\vspace*{-24pt}
\begin{align}\label{eq:gamma:parameterization:2}
	\text{Shape: } k>0			&&	\text{Scale: }\theta>0
\end{align}

In the second parameterization, the scale parameter $\theta$ corresponds to the inverse of the rate parameter $\lambda$,
allowing flexibility in the modelization of the distribution. Indeed, the application of the scale parameter is
pertinent when one wants to use the value of the mean time between events in the probability density function, as
opposed to the total time until a given number of events occurs.

That being said, in this document, as the two parameterizations only exist for the sake of convenience and are identical
in their results, only the parameterization 1 presented in \autoref{eq:gamma:parameterization:1} will be considered and
used for proofs.

\pagebreak
\section{Properties to prove}

\subsection{Support}
\begin{equation*}
	x\in (0,\,\infty)
\end{equation*}

\subsection{Probability density function}\label{section:gamma:pdf}
\begin{equation*}\label{eq:gamma:pdf:2}
	f(x) = \frac{\lambda^\alpha}{\Gamma(\alpha)}x^{\alpha-1}e^{-\lambda x}
\end{equation*}
\begin{proof}
	The probability density function of the gamma distribution will be proved using induction. First, take
	\begin{equation*}
		W=\sum^n{i=1}X_i
	\end{equation*}
	where $W$ is a sum of $n$ i.i.d. random variables following the exponential distribution $\E\text{xp}(\lambda)$.

	\noindent For $n=1$:
	\begin{equation*}
		\begin{split}
			W		&=	X_1\\
			f(x)	&=	\lambda e^{\lambda x}\\
			M(t)	&=	\frac{\lambda}{\lambda - t}
		\end{split}
	\end{equation*}
	Then, with the induction that $n=k$ works such that:
	\begin{equation*}
		\begin{split}
			W							&=	\sum^\alpha_{i=1}X_i\\
			M_{\sum^\alpha_{i=1}X_i}(t)	&=	E(e^{t\sum^\alpha_{i=1}X_i})\\
										&=	E(\prod^\alpha_{i=1}e^{tX_i})\\
										&=	\prod^\alpha_{i=1}E(e^{tX_i})\\
										&=	\prod^\alpha_{i=1}M_{X_i}(t)\\
										&=	\left(\frac{\lambda}{\lambda - t}\right)^\alpha
		\end{split}
	\end{equation*}
	Here is the proof that $n=k+1$ works such that:
	\begin{equation*}
		\begin{split}
			W	&=	\sum^{\alpha+1}_{i=1}X_i\\
				&=	\sum^{\alpha}_{i=1}\left[X_i\right] + X_{\alpha+1}
		\end{split}
	\end{equation*}
	\begin{equation*}
		\begin{split}
			M_{\sum^\alpha_{i=1}\left[X_i\right] + X_{\alpha+1}}(t)		&=	E\left(e^{t\left(\sum^\alpha_{i=1}\left[X_i\right] + X_{\alpha+1}\right)}\right)\\
																		&=	E\left(e^{t\left(\sum^\alpha_{i=1}X_i\right)}e^{tX_{\alpha+1}}\right)\\
																		&=	E\left(e^{t\left(\sum^\alpha_{i=1}X_i\right)}\right)E\left(e^{tX_{\alpha+1}}\right)\\
																		&=	M_{\sum^\alpha_{i=1}X_i}(t)M(t)\\
																		&=	\left(\frac{\lambda}{\lambda - t}\right)^\alpha\left(\frac{\lambda}{\lambda - t}\right)\\
																		&=	\left(\frac{\lambda}{\lambda - t}\right)^{\alpha+1}
		\end{split}
	\end{equation*}
	\bigred{What to do next? I'm not yet finding f(x)}
\end{proof}

\subsection{Cumulative distribution function}
\begin{equation*}\label{eq:gamma-cdf-2}
	F(x) = \frac{1}{\Gamma(\alpha)}\gamma(\alpha,\, \lambda x)
\end{equation*}

\subsection{Expected Value}
The expected value is also known as the theoretical mean:
\begin{equation*}\label{eq:gamma-expected-value-2}
	\mu = E(x) = \frac{\alpha}{\lambda}
\end{equation*}
\begin{proof}
	Using the moment generating function obtained in \autoref{section:gamma:pdf}, the expected value of a gamma
	distribution may be found.
	\begin{equation*}
		\begin{split}
			E(x)		&=	M^\prime(0)\\
			M^\prime(t)	&=	\deriv{t}\left(\frac{\lambda}{\lambda - t}\right)^\alpha\\
						&=	\lambda\alpha\left(\frac{\lambda}{\lambda - t}\right)^{\alpha-1}\deriv{t}\left(\frac{1}{\lambda - t}\right)\\
						&=	\frac{\lambda\alpha}{(\lambda - t)^2}\left(\frac{\lambda}{\lambda - t}\right)^{\alpha-1}\\
			M^\prime(0)	&=	\frac{\lambda\alpha}{\lambda^2}\left(\frac{\lambda}{\lambda}\right)^{\alpha-1}\\
						&=	\frac{\alpha}{\lambda}\\
		\end{split}
	\end{equation*}
\end{proof}

\subsection{Median}
\noindent There is no simple closed form equation for the median of a gamma distribution.
\bigred{EXPLAIN WHY}

\subsection{Mode}
\begin{equation*}\label{eq:gamma-mode-2}
	\text{Mode} = \frac{(\alpha-1)}{\lambda}\text{ for }\alpha \geq 1
\end{equation*}

\subsection{Variance}
\begin{equation*}\label{eq:gamma-variance-2}
	\text{Var}(x) = \frac{\alpha}{\lambda^2}
\end{equation*}
\begin{proof}
	Using the moment generating function obtained in \autoref{section:gamma:pdf}, the variance value of a gamma
	distribution may be found.
	\begin{equation*}
		\begin{split}
			E(x)	&=	M^{\prime\prime}(0)\\
		\end{split}
	\end{equation*}
\end{proof}

\subsection{Skewness}
\begin{equation*}\label{eq:gamma-skewness-2}
	\text{Skewness} = \frac{2}{\sqrt{\alpha}}
\end{equation*}

\subsection{Excess kurtosis}
\begin{equation*}\label{eq:gamma-kurtosis-2}
	\text{Kurtosis} = \frac{6}{\alpha}
\end{equation*}

\subsection{Entropy}
\begin{equation*}\label{eq:gamma-entropy-2}
	\text{Entropy} = \alpha + \ln{\lambda} + \ln{\Gamma(\alpha)} + (1-\alpha)\psi(\alpha)
\end{equation*}

\subsection{Moment generating function}
\begin{equation*}\label{eq:gamma-mgf-2}
	M(t) = \left(1-\frac{t}{\lambda}\right)^{-\alpha}\text{ for }t < \lambda
\end{equation*}

\subsection{Characteristic function}
\begin{equation*}\label{eq:gamma-cf-2}
	\text{CF} = \left(1-\frac{it}{\lambda}\right)^{-\alpha}
\end{equation*}

\subsection{Methods of moments}
\begin{equation*}\label{eq:gamma-mom-2}
	\begin{split}
		\alpha	&=	\frac{E(X)^2}{\text{Var}(X)}\\
		\lambda	&=	\frac{E(X)}{\text{Var}(X)}
	\end{split}
\end{equation*}
%======================================================================================================================
% END OF DOCUMENT =====================================================================================================
%======================================================================================================================
\pagebreak

\begin{appendix}
	\phantomsection\listoffigures
	\phantomsection\listoftables
\end{appendix}

\pagebreak\phantomsection\printbibliography[heading=bibintoc,title={References}]
\end{document}
