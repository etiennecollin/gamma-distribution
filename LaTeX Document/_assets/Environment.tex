% DEFINITIONS =========================================================================================================
% Paths
\addbibresource{\basepath/bibliography.bib} %Imports bibliography file
\graphicspath{{images/}}

% lstlistings setup ---------------------------------------------------------------------------------------------------
\definecolor{commentsColor}{gray}{0.6}
\definecolor{keywordsColor}{RGB}{249,38,114}
\definecolor{stringColor}{RGB}{230, 219, 116}
\definecolor{mygreen}{rgb}{0,0.6,0}
\definecolor{mygray}{rgb}{0.5,0.5,0.5}
\definecolor{mymauve}{rgb}{0.58,0,0.82}
\lstset{ 
  backgroundcolor=\color{white},   % choose the background color; you must add \usepackage{color} or \usepackage{xcolor}; should come as last argument
  basicstyle=\footnotesize,        % the size of the fonts that are used for the code
  breakatwhitespace=false,         % sets if automatic breaks should only happen at whitespace
  breaklines=true,                 % sets automatic line breaking
  captionpos=b,                    % sets the caption-position to bottom
  commentstyle=\color{mygreen},    % comment style
  deletekeywords={},            % if you want to delete keywords from the given language
  escapeinside={\%*}{*)},          % if you want to add LaTeX within your code
  extendedchars=true,              % lets you use non-ASCII characters; for 8-bits encodings only, does not work with UTF-8
  firstnumber=0,                % start line enumeration with line 0
  frame=single,	                   % adds a frame around the code
  keepspaces=true,                 % keeps spaces in text, useful for keeping indentation of code (possibly needs columns=flexible)
  keywordstyle=\color{blue},       % keyword style
  language=Octave,                 % the language of the code
  morekeywords={*},            % if you want to add more keywords to the set
  numbers=left,                    % where to put the line-numbers; possible values are (none, left, right)
  numbersep=5pt,                   % how far the line-numbers are from the code
  numberstyle=\tiny\color{mygray}, % the style that is used for the line-numbers
  rulecolor=\color{black},         % if not set, the frame-color may be changed on line-breaks within not-black text (e.g. comments (green here))
  showspaces=false,                % show spaces everywhere adding particular underscores; it overrides 'showstringspaces'
  showstringspaces=false,          % underline spaces within strings only
  showtabs=false,                  % show tabs within strings adding particular underscores
  stepnumber=2,                    % the step between two line-numbers. If it's 1, each line will be numbered
  stringstyle=\color{mymauve},     % string literal style
  tabsize=4,	                   % sets default tabsize to 2 spaces
  title=\lstname                   % show the filename of files included with \lstinputlisting; also try caption instead of title
}

% Sections alignment --------------------------------------------------------------------------------------------------
% Gets label of current section
\makeatletter\newcommand{\currentlabel}{\@currentlabelname}\makeatother
% Sets alignment of sections
\newcommand{\formatsection}[1]{
    \ifthenelse{\equal{#1}{center}}{
		%Makes sections centered
		\titleformat{\section}{\normalfont\Large\bfseries\centering}{\thesection}{1em}{}
    }{
        \ifthenelse{\equal{#1}{right}}{
			%Makes sections right-aligned
			\titleformat{\section}{\normalfont\Large\bfseries\raggedleft}{\thesection}{1em}{}
        }{
			\ifthenelse{\equal{#1}{left}}{
				%Makes sections left-aligned
				\titleformat{\section}{\normalfont\Large\bfseries\raggedright}{\thesection}{1em}{}
        	}{
        	}
        }
    }
}

% tabularx
\newcolumntype{Y}{>{\centering\arraybackslash}X}

% Caption/Ref settings ----------------------------------------------------------------------------------------------------
\captionsetup{
    format = hang,
    labelfont = sc,
	skip = 6pt,
	hypcapspace = 96pt
}

% TOC -----------------------------------------------------------------------------------------------------------------
% Sets TOC depth and change changes its display name
\setcounter{secnumdepth}{0}
\addto\captionsenglish{
  \renewcommand{\contentsname}{Table of Contents}
}

% SIunitx -------------------------------------------------------------------------------------------------------------
% Sets \per to display as frac
\sisetup{per-mode=symbol-or-fraction}
\DeclareSIUnit\Molar{\mole\per\liter}
\DeclareSIUnit\molar{\textsc{M}}
\DeclareSIUnit\clight{\text{\ensuremath{c}}_{0}}

% listings code highlighting ------------------------------------------------------------------------------------------
\lstset{ %
  backgroundcolor=\color{white},				% choose the background color; you must add \usepackage{color} or \usepackage{xcolor}
  basicstyle=\footnotesize,						% the size of the fonts that are used for the code
  breakatwhitespace=false,						% sets if automatic breaks should only happen at whitespace
  breaklines=true,								% sets automatic line breaking
  captionpos=b,									% sets the caption-position to bottom
  commentstyle=\color{commentsColor}\textit,	% comment style
  deletekeywords={...},							% if you want to delete keywords from the given language
  escapeinside={\%*}{*)},						% if you want to add LaTeX within your code
  extendedchars=true,							% lets you use non-ASCII characters; for 8-bits encodings only, does not work with UTF-8
  frame=tb,										% adds a frame around the code
  keepspaces=true,								% keeps spaces in text, useful for keeping indentation of code (possibly needs columns=flexible)
  keywordstyle=\color{keywordsColor}\bfseries,	% keyword style
  language=Python,								% the language of the code (can be overdid per snippet)
  otherkeywords={*,...},						% if you want to add more keywords to the set
  numbers=left,									% where to put the line-numbers; possible values are (none, left, right)
  numbersep=5pt,								% how far the line-numbers are from the code
  numberstyle=\tiny\color{commentsColor},		% the style that is used for the line-numbers
  rulecolor=\color{black},						% if not set, the frame-color may be changed on line-breaks within not-black text (e.g. comments (green here))
  showspaces=false,								% show spaces everywhere adding particular underscores; it overrides 'showstringspaces'
  showstringspaces=false,						% underline spaces within strings only
  showtabs=false,								% show tabs within strings adding particular underscores
  stepnumber=1,									% the step between two line-numbers. If it's 1, each line will be numbered
  stringstyle=\color{stringColor},				% string literal style
  tabsize=2,									% sets default tabsize to 2 spaces
  title=\lstname,								% show the filename of files included with \lstinputlisting; also try caption instead of title
  columns=fixed									% Using fixed column width (for e.g. nice alignment)
}

% BibLaTex ------------------------------------------------------------------------------------------------------------
% BibLaTex definitions
\let\oldcite\cite
\renewcommand{\cite}{\autocite}
\let\oldsupercite\supercite
\renewcommand{\supercite}[1]{\textsuperscript{[}\oldsupercite{#1}\textsuperscript{]}}
% Makes number created by \supercite to be between brackets.
\DeclareCiteCommand{\supercite}[\mkbibsuperscript]{%
	\iffieldundef{prenote}{}{\BibliographyWarning{Ignoring prenote argument}}%
	\iffieldundef{postnote}{}{\BibliographyWarning{Ignoring postnote argument}}%
}
{\usebibmacro{citeindex}\,\bibopenbracket\usebibmacro{cite}\bibclosebracket}
{\supercitedelim}{}

% BASIC DOCUMENT SETTINGS AND TITLE PAGE ==============================================================================
% PDF export settings
\hypersetup{
	pdftex,
	pdfauthor = {Etienne Collin},
	bookmarksnumbered = true,
	bookmarksopen = true,
	bookmarksopenlevel = 1,
	pdfstartview = Fit,
	pdfpagemode = UseOutlines,
	pdfpagelayout = TwoPageRight,
	colorlinks = true,		% Colors links instead of ugly boxes
	urlcolor = DarkBlue,	% Color for external hyperlinks
	linkcolor = black,		% Color of internal links
	citecolor = black,		% Color of citations
	filecolor = black,		% Color of file links
}

% ENVIRONMENTS ========================================================================================================
% hmwkProblem ---------------------------------------------------------------------------------------------------------
\newcounter{partCounter}
\renewcommand{\part}[1]{\noindent\textbf{\large Part (\alph{partCounter})}\stepcounter{partCounter}\\}
\newcounter{hmwkProblemCounter}
\setcounter{hmwkProblemCounter}{1}

% Continued Header Changer
\newcommand{\enterProblemHeader}[1]{
	\nobreak\extramarks{}{}\nobreak{}
	\nobreak\extramarks{Problem \arabic{#1} (continued)}{Problem \arabic{#1} continued on next page\ldots}\nobreak{}
}
\newcommand{\exitProblemHeader}[1]{
	\nobreak\extramarks{Problem \arabic{#1} (continued)}{Problem \arabic{#1} continued on next page\ldots}\nobreak{}
	\stepcounter{#1}
	\nobreak\extramarks{}{}\nobreak{}
}

% Environment. Optional argument will adjust the problem counter.
\newenvironment{hmwkProblem}[1][-1]{
	\ifnum#1>0
		\setcounter{hmwkProblemCounter}{#1}
	\fi
	\pagebreak
	\section{Problem \arabic{hmwkProblemCounter}}
	\stepcounter{section}
	\setcounter{partCounter}{1}
	\enterProblemHeader{hmwkProblemCounter}
}{
	\exitProblemHeader{hmwkProblemCounter}
}

% sect ----------------------------------------------------------------------------------------------------------------
% Will write "continued" when content spans >1 page. Argument adjusts the name of the section.
\newenvironment{sect}[1]{
	\pagebreak
	\section{#1}
	\stepcounter{section}
	\setcounter{partCounter}{1}
	\nobreak\extramarks{}{}\nobreak{}
	\nobreak\extramarks{\currentlabel\ (continued)}{\currentlabel\ continued on next page\ldots}\nobreak{}
}{
	\nobreak\extramarks{\currentlabel\ (continued)}{\currentlabel\ continued on next page\ldots}\nobreak{}
	\nobreak\extramarks{}{}\nobreak{}
}

% SETUP/ALIASES =======================================================================================================
\newcommand{\imagecenter}[2][1]{
	\begin{center}
		\includegraphics[width=\textwidth*{#1}]{#2}
	\end{center}
}
\newcommand{\imageleft}[2][1]{
	\begin{flushleft}
		\includegraphics[width=\textwidth*{#1}]{#2}
	\end{flushleft}
}
\newcommand{\imageright}[2][1]{
	\begin{flushright}
		\includegraphics[width=\textwidth*{#1}]{#2}
	\end{flushright}
}
\newcommand{\figurecenter}[4]{
	\begin{figure}[H]
		\centering
		\begin{minipage}[H]{#1\textwidth}
			\centering
			\includegraphics[width=\textwidth]{#2}
			\caption{#3}
			\label{#4}
		\end{minipage}
	\end{figure}
}
\newcommand{\figuredual}[6]{
	\begin{figure}[H]
		\begin{minipage}[H]{0.48\textwidth}
			\centering
			\includegraphics[width=\textwidth]{#1}
			\raggedleft
			\caption{#2}
			\label{#3}
		\end{minipage}
		\hfill
		\begin{minipage}[H]{0.48\textwidth}
			\centering
			\includegraphics[width=\textwidth]{#4}
			\caption{#5}
			\label{#6}
		\end{minipage}
	\end{figure}
}
\newcommand{\figuredualsingle}[6]{
	\begin{figure}[H]
		\begin{minipage}[H]{0.48\textwidth}
			\centering
			\includegraphics[width=\textwidth]{#1}
			\subcaption{#2}\label{#6-a}
		\end{minipage}
		\hfill
		\begin{minipage}[H]{0.48\textwidth}
			\centering
			\includegraphics[width=\textwidth]{#3}
			\subcaption{#4}\label{#6-b}
		\end{minipage}
		\caption{#5}\label{#6}
	\end{figure}
}

% Theorems
\theoremstyle{plain}
\newtheorem{definition}{Definition}[section]
\newtheorem{theorem}{Theorem}[section]
\newtheorem{example}{Example}[theorem]
\newtheorem{corollary}{Corollary}[theorem]
\newtheorem{lemma}{Lemma}[theorem]
\newtheorem{proposition}{Proposition}[theorem]
% Unnumbered version
\newtheorem*{definition*}{Definition}
\newtheorem*{theorem*}{Theorem}
\newtheorem*{example*}{Example}
\newtheorem*{corollary*}{Corollary}
\newtheorem*{lemma*}{Lemma}
\newtheorem*{proposition*}{Proposition}

\theoremstyle{definition}
\newtheorem{exercise}{Exercise}[section]
% Unnumbered version
\newtheorem*{exercise*}{Exercise}


\theoremstyle{remark}
\newtheorem{remark}{Remark}
\newtheorem{case}{Case}
\newtheorem{claim}{Claim}[section]
% Unnumbered version
\newtheorem*{remark*}{Remark}
\newtheorem*{case*}{Case}
\newtheorem*{claim*}{Claim}

\renewcommand\qedsymbol{QED}

% Useful shortcuts
\newcommand{\alg}[1]{\textsc{\bfseries\footnotesize #1}}
\newcommand{\deriv}[2][]{\mathop{}\!\frac{\mathrm{d}^{#1}}{\mathrm{d}#2^{#1}}}	% For derivatives
\newcommand{\pderiv}[2][]{\frac{\partial^{#1}}{\partial#2^{#1}}}				% For partial derivatives
\newcommand{\diff}[2][]{\mathop{}\!\mathrm{d^{#1}}#2}							% Integral dx

% Enables new list environment that supports nesting
\newlist{legal}{enumerate}{10}
\setlist[legal]{label*=.\arabic*}
\setlist[legal,1]{label=\arabic*}

% Temporary text markers
\newcommand{\solution}{\noindent\textbf{\large Solution}}				% Alias for the Solution section header
\newcommand{\missing}{\textcolor{red}{\textbf{VALUE MISSING}}}	% Creates a big red "Missing Value" string.
\newcommand{\inprogress}{\textcolor{red}{\textbf{IN PROGRESS}}}	% Creates a big red "In Progress" string.
\newcommand{\incomplete}{\textcolor{red}{\textbf{INCOMPLETE}}}	% Creates a big red "Incompletes" string.

% END OF SETUP DOCUMENT ===============================================================================================